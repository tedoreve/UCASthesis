\chapter{关于超新星遗迹的前身星星风如何影响其射电形态}
\label{SW}

\section{研究历史及意义}
\label{SWintro}
数值模拟是帮助我们理解超新星遗迹演化的重要手段,随着集群计算能力的提升,我们可以做到很多
之前难以做到的模拟。
一开始,对于遗迹的模拟更多是一维流体模拟结合分析解来定性解释遗迹的一些观测特征。
接着,逐渐有很多非常不错的二维流体模拟的工作涌现,尝试研究一些解析解难以描述的磁场放大、
弥漫激波加速和不稳定性\citep{Jun1996,Kang2006,Fang2012}。
而最近,我们已经可以很好的进行三位次流体模拟工作,并将模拟结果转化为射电、光学和X射线图像
以方便与观测相比较\citep{Orlando2007,Meyer2015,Zhang2017},从而帮助我们解释更多观测。
比如,\citet{Orlando2007}注意到有一类遗迹只有单边壳层,然而大部分超新星爆发都可看作球对称,
这是很难通过解析解解释的,于是他们通过假设带周围介质有密度或者磁场梯度的初始条件模拟这一
类遗迹,不过没有解释为什么会存在这样的梯度。
我们在章节~\ref{W51C}也采用了这种设置,但是根据观测密度梯度几乎不可能存在,所以我们
采取假设磁场梯度。


在之前对超新星遗迹W51C的模拟中,我们发现初始的介质分布对模拟结果有很大的影响。
而实际上


\section{模拟模型}
\label{SWmod}

\section{结果和讨论}
\label{SWres}

\section{总结}
\label{SWsum}
