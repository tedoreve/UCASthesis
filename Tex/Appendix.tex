\chapter{基本代码}
\label{Basic}

\section{definition.h}
\label{def}
\begin{lstlisting}
#define  PHYSICS                 MHD
#define  DIMENSIONS              3
#define  COMPONENTS              3
#define  GEOMETRY                CARTESIAN
#define  BODY_FORCE              NO
#define  COOLING                 NO
#define  RECONSTRUCTION          LINEAR
#define  TIME_STEPPING           RK2
#define  DIMENSIONAL_SPLITTING   NO
#define  NTRACER                 0
#define  USER_DEF_PARAMETERS     14

/* -- physics dependent declarations -- */

#define  EOS                     IDEAL
#define  ENTROPY_SWITCH          NO
#define  DIVB_CONTROL            CONSTRAINED_TRANSPORT
#define  BACKGROUND_FIELD        NO
#define  RESISTIVITY             NO
#define  THERMAL_CONDUCTION      NO
#define  VISCOSITY               NO
#define  ROTATING_FRAME          NO

/* -- user-defined parameters (labels) -- */

#define  E_EJ                    0
#define  M_EJ                    1
#define  R_EJ                    2
#define  N_H                     3
#define  U_AM                    4
#define  S_PI                    5
#define  N_PI                    6
#define  W_C                     7
#define  BMAG                    8
#define  THETA                   9
#define  PHI                     10
#define  GAMMA                   11
#define  Temp                    12
#define  NU_VISC                 13

/* [Beg] user-defined constants (do not change this line) */

#define  UNIT_DENSITY            CONST_mp
#define  UNIT_LENGTH             CONST_pc
#define  UNIT_VELOCITY           1.0e9
#define  ADD_TURBULENCE          YES

/* [End] user-defined constants (do not change this line) */

/* -- supplementary constants (user editable) -- */

#define  INITIAL_SMOOTHING         NO
#define  WARNING_MESSAGES          NO
#define  PRINT_TO_FILE             NO
#define  INTERNAL_BOUNDARY         NO
#define  SHOCK_FLATTENING          NO
#define  CHAR_LIMITING             YES
#define  LIMITER                   VANLEER_LIM
#define  CT_EMF_AVERAGE            ARITHMETIC
#define  CT_EN_CORRECTION          YES
#define  ASSIGN_VECTOR_POTENTIAL   YES
#define  UPDATE_VECTOR_POTENTIAL   NO

\end{lstlisting}

\section{init.c}
\label{init}
\begin{lstlisting}

  #include "pluto.h"
  #include "math.h"
  #include "time.h"
  #include "stdlib.h"
  #include "stdio.h"
  /* ********************************************************************* */
  void Init (double *us, double x1, double x2, double x3)
  /*
   *
   *
   *
   *********************************************************************** */
  {
    static int first_call = 1;
    double r, theta, phi, B0, E_ej, M_ej, R_ej, n_h, w_c, s, n, n_ISM, r_c, T, u, g_gamma;
    double fnc, alpha, v_ej, t, rho_ch, R_ch, eta, ph, l, up, down;
    E_ej    = g_inputParam[E_EJ];         //爆发能量
    M_ej    = g_inputParam[M_EJ];         //爆发质量
    R_ej    = g_inputParam[R_EJ];         //爆发半径
    n_h     = g_inputParam[N_H];          //氢数密度
    u       = g_inputParam[U_AM];            //介质总体数密度,U是平均分子权重
    w_c     = g_inputParam[W_C];          //激波区质量与爆发总质量之比
    n       = g_inputParam[N_PI];            //激波区密度幂指数
    s       = g_inputParam[S_PI];            //激波区速度幂指数
    n_ISM   = n_h*u;                     //激波前均匀介质区数密度
    T       = g_inputParam[Temp];         //初始温度
    g_gamma = g_inputParam[GAMMA];        //绝热系数

    r_c     = R_ej*w_c;
    fnc     = 3.0/4.0/CONST_PI*(1.0-n/3.0)/(1.0-n/3.0*pow(w_c,3.0-n));
    alpha   = (3.0-n)/(5.0-n)*(pow(w_c,n-5.0)-n/5.0)/(pow(w_c,n-3.0)-n/3.0)*pow(w_c,2);
    v_ej    = pow(E_ej/(M_ej*alpha*0.5),0.5);
    t       = R_ej/v_ej;
    rho_ch  = M_ej/pow(R_ej,3);
    R_ch    = pow(M_ej,1.0/3.0)*pow(n_ISM,-1.0/3.0);

    ph      = 1.1;  //n=0
    l       = 0.343;
  //  up      = 1 + (n-3)/3*pow(phi/l*fnc,0.5)*pow(R_ej,1.5)
  //  down    = 1 + (n/3)*pow(phi/l*fnc,0.5)*pow(R_ej,1.5)
  //  eta     = up/down

  //  rho_c   = (1-eta)*M_ej/(4/3*CONST_PI*pow(r_c,3)); //激波后均匀介质区密度
  //  C       = 2.0*CONST_PI*pow(r_c,n)*rho_c*pow(R_ej,-2*s);
  //  index   = 2*s+3-n;
  //
  //  part1   = C/pow(r_c,n)*pow(r_c,2*s+3)/(2*s+3);
  //  part2   = C*(pow(R_ej,index)-pow(r_c,index))/index;
  //  v0      = pow(E_ej,0.5)*pow(part1+part2,-0.5);

    r = D_EXPAND(x1*x1, + x2*x2, + x3*x3);
    r = sqrt(r);

    us[RHO] = n_ISM;
    us[VX1] = 0.0;
    us[VX2] = 0.0;
    us[VX3] = 0.0;
    us[PRS] = n_ISM*CONST_kB*T/1.67e-6;

    #if ADD_TURBULENCE == YES
    if (first_call){
      int k, input_var[200];
      for (k = 0; k< 200; k++) input_var[k] = -1;
      input_var[0] = RHO;
      input_var[1] = BX1;
      input_var[2] = BX2;
      input_var[3] = BX3;
      input_var[4] = -1;
      InputDataSet ("./grid0.out",input_var);
      InputDataRead("./rho0.dbl"," ");
      first_call = 0;
    }
    InputDataInterpolate(us, x1, x2, x3);  /* -- interpolate density from
                                                input data file -- */
    #endif
  /*
    if (r > 2.5 && r <= 3)
    {
      us[RHO] = 60*pow(r/r_c,0.0);
    }
  */

    if (r <= r_c && r != 0)
    {
      up      = 1.0 + (n-3.0)/3.0*pow(ph/l*fnc,0.5)*pow(r/R_ch,1.5);
      down    = 1.0 + (n/3.0)*pow(ph/l*fnc,0.5)*pow(r/R_ch,1.5);
      eta     = up/down;
  //    eta     = 1.0;
      us[RHO] = rho_ch*fnc*pow(w_c,-n);
      us[VX1] = (x1/t)*eta;
      us[VX2] = (x2/t)*eta;
      us[VX3] = (x3/t)*eta;
      us[PRS] = rho_ch*fnc*pow(w_c,-n)*CONST_kB*T/1.67e-6;
    }

    if (r >  r_c && r <= R_ej)
    {
     // us[RHO] = a[(int) fabs(x1*x2*100)]*rho_c*pow(r/r_c,-n);
      up      = 1.0 + (n-3.0)/3.0*pow(ph/l*fnc,0.5)*pow(r/R_ch,1.5);
      down    = 1.0 + (n/3.0)*pow(ph/l*fnc,0.5)*pow(r/R_ch,1.5);
      eta     = up/down;
  //    eta     = 1.0;
      us[RHO] = rho_ch*fnc*pow(r/R_ej,-n);
      us[VX1] = (x1/t)*eta;
      us[VX2] = (x2/t)*eta;
      us[VX3] = (x3/t)*eta;
      us[PRS] = rho_ch*fnc*pow(r/R_ej,-n)*CONST_kB*T/1.67e-6;
    }

  //  printf("%e\n",eta);
    //theta = g_inputParam[THETA]*CONST_PI/180.0;
    //phi   =   g_inputParam[PHI]*CONST_PI/180.0;
    B0    = g_inputParam[BMAG];

    //us[BX1] = B0*sin(theta)*cos(phi);
    //us[BX2] = B0*sin(theta)*sin(phi);
    //us[BX3] = 0.0;

    #if GEOMETRY == CARTESIAN
     us[AX1] = 0.0;
     us[AX2] =  us[BX3]*x1;
     us[AX3] = -us[BX2]*x1 + us[BX1]*x2;
    #elif GEOMETRY == CYLINDRICAL
     us[AX1] = us[AX2] = 0.0;
     us[AX3] = 0.5*us[BX2]*x1;
    #endif

    #if BACKGROUND_FIELD == YES
     us[BX1] = us[BX2] = us[BX3] =
     us[AX1] = us[AX2] = us[AX3] = 0.0;
    #endif

  }
  /* ********************************************************************* */
  void Analysis (const Data *d, Grid *grid)
  /*
   *
   *
   *********************************************************************** */
  {

  }
  /* ********************************************************************* */
  void UserDefBoundary (const Data *d, RBox *box, int side, Grid *grid)
  /*
   *
   *********************************************************************** */
  {
  }
  #if BACKGROUND_FIELD == YES
  /* ********************************************************************* */
  void BackgroundField (double x1, double x2, double x3, double *B0)
  /*!
   * Define the component of a static, curl-free background
   * magnetic field.
   *
   *********************************************************************** */
  {
  /*
    static int first_call = 1;
    double theta, phi;
    static double sth,cth,sphi,cphi;

    if (first_call){
      theta = g_inputParam[THETA]*CONST_PI/180.0;
      phi   =   g_inputParam[PHI]*CONST_PI/180.0;
      sth   = sin(theta);
      cth   = cos(theta);
      sphi  = sin(phi);
      cphi  = cos(phi);
      first_call = 0;
    }
    EXPAND(B0[IDIR] = g_inputParam[BMAG]*sth*cphi; ,
           B0[JDIR] = g_inputParam[BMAG]*sth*sphi; ,
           B0[KDIR] = g_inputParam[BMAG]*cth;)

  /*
    theta = g_inputParam[THETA]*CONST_PI/180.0;
    phi   =   g_inputParam[PHI]*CONST_PI/180.0;

    B0[IDIR] = g_inputParam[BMAG]*sin(theta)*cos(phi);
    B0[JDIR] = g_inputParam[BMAG]*sin(theta)*sin(phi);
    B0[KDIR] = g_inputParam[BMAG]*cos(theta);
  */

  }
  #endif

\end{lstlisting}

\section{pluto.ini}
\label{plu}

\begin{lstlisting}
[Grid]

X1-grid    1    -37.5    256    u    37.5
X2-grid    1    -37.5    256    u    37.5
X3-grid    1    -37.5    256    u    37.5

[Chombo Refinement]

Levels           4
Ref_ratio        2 2 2 2 2
Regrid_interval  2 2 2 2
Refine_thresh    0.3
Tag_buffer_size  3
Block_factor     4
Max_grid_size    32
Fill_ratio       0.75

[Time]

CFL              0.4
CFL_max_var    1.1
tstop            300.0
first_dt         1.e-2

[Solver]

Solver         hll

[Boundary]

X1-beg        outflow
X1-end        outflow
X2-beg        outflow
X2-end        outflow
X3-beg        outflow
X3-end        outflow

[Static Grid Output]

uservar    0
dbl        60  -1   single_file
flt       -1.0  -1   single_file
vtk       -1.0  -1   single_file
tab       -1.0  -1
ppm       -1.0  -1
png       -1.0  -1
log        5
analysis  -1.0  -1

[Chombo HDF5 output]

Checkpoint_interval  -1.0  0
Plot_interval         1.0  0

[Parameters]

E_EJ                 26.45
M_EJ                 445.25
R_EJ                 4
N_H                  0.21
U_AM                 1.3
S_PI                 1.0
N_PI                 0
W_C                  0.1
BMAG                 2.0e-3
THETA                90.0
PHI                  90.0
GAMMA                1.7
Temp                 1.0e2
NU_VISC              3.87e-9

\end{lstlisting}


\chapter{工具型代码}
\label{Code}

\section{单位制计算}
\label{Codeu}


\begin{lstlisting}
from astropy import units as un
from astropy import constants as con
import numpy as np

#====================单位计算====================

UNIT_DENSITY = 1*con.m_p/un.cm**3
UNIT_LENGTH  = 1*un.pc
UNIT_VELOCITY= 1e4*un.km/un.s
UNIT_B = (UNIT_VELOCITY*np.sqrt(4*np.pi*UNIT_DENSITY)).to(un.g**0.5*un.cm**-0.5*un.s**-1).value*un.G
UNIT_t = UNIT_LENGTH/UNIT_VELOCITY
UNIT_P = UNIT_DENSITY*UNIT_VELOCITY**2
UNIT_M = UNIT_DENSITY*UNIT_LENGTH**3
UNIT_E = UNIT_M*UNIT_VELOCITY**2
#UNIT_B = ((UNIT_E/UNIT_LENGTH**3)**0.5).value*un.G
UNIT_NU= UNIT_P*UNIT_t
UNIT_G = (UNIT_VELOCITY/UNIT_LENGTH)**2/UNIT_DENSITY

#===============输入需要转换的参量================

n   = 0.21*con.m_p/un.cm**3
l   = 4*un.pc
v   = 490*un.km/un.s
B   = 9*un.uG
t   = 1000*un.yr
P   = 1*un.Ba
E_th= 0.96*un.erg
E   = 2.0e51*un.erg
M   = 15.9*con.M_sun
nu  = 2*un.uPa*un.s
G   = 1*con.G

#=================开始转换========================

n   /= UNIT_DENSITY
l   /= UNIT_LENGTH
v   /= UNIT_VELOCITY
B   /= UNIT_B
t   /= UNIT_t
P   /= UNIT_P
E   /= UNIT_E
M   /= UNIT_M
nu  /= UNIT_NU
G   /= UNIT_G

#================输出结果=========================

print('n = ', n.to('').value, '\n'
      'l = ', l.to('').value, '\n'
      'v = ', v.to('').value, '\n'
      'B = ', B.to('').value, '\n'
      't = ', t.to('').value, '\n'
      'P = ', P.to('').value, '\n'
      'E = ', E.to('').value, '\n'
      'M = ', M.to('').value, '\n'
      'nu = ', nu.to('').value, '\n'
      'G = ', G.to('').value, '\n'
      )

\end{lstlisting}


\section{构造初始背景}
\label{Codeb}

\begin{lstlisting}

import numpy as np
import time as ti
from astropy.io import fits

#=====================加入星风==================

def stellar_wind(wdir,number):
    import pyPLUTO as pp
    pp.nlast_info(w_dir=wdir)
    D = pp.pload(number,w_dir=wdir)
    print(D.rho.shape)

    rho = D.rho
    bx1 = D.bx1
    bx2 = D.bx2
    bx3 = D.bx3
    vx1 = D.vx1
    vx2 = D.vx2
    vx3 = D.vx3

    rho = np.transpose(rho)
    bx1 = np.transpose(bx1)
    bx2 = np.transpose(bx2)
    bx3 = np.transpose(bx3)
    vx1 = np.transpose(vx1)
    vx2 = np.transpose(vx2)
    vx3 = np.transpose(vx3)

    return rho,bx1,bx2,bx3,vx1,vx2,vx3

#==================加入磁场====================

def toff(f):
    def wrapper(*args):
        start = ti.time()
        f(*args)
        end   = ti.time()
        print(end-start)
    return wrapper

def f(i,j,k):
    return i*2+j*2

#@toff
def magnetism(width):
    x = np.fromfunction(f,(width,width,width))/500000
    x = np.rot90(x,k=1)
    x = np.transpose(x)
    x = np.reshape(x,width**3,1)
    x = x + 0.001
    return x*0, x, x

#=================组合背景=====================

def combine(components,infilename,outfilename,width,index,rho_constant,sw,clump,mag):
    if 'sw' in components:
        wdir,number = sw
        rho,bx1,bx2,bx3,vx1,vx2,vx3    = stellar_wind(wdir,number)

        rho       = np.reshape(rho,(1,width**3))
        bx1       = np.reshape(bx1,(1,width**3))
        bx2       = np.reshape(bx2,(1,width**3))
        bx3       = np.reshape(bx3,(1,width**3))
        vx1       = np.reshape(vx1,(1,width**3))
        vx2       = np.reshape(vx2,(1,width**3))
        vx3       = np.reshape(vx3,(1,width**3))

        total     = np.concatenate((rho,bx1,bx2,bx3,vx1,vx2,vx3))

    total     = total.astype(float)
    total.tofile(outfilename)

    return total

#==================网格定义========================

def grid(outfilename,ra,width):
    b=np.linspace(-ra,ra,width+1)
    c=np.linspace(1,width,width)
    f=open(outfilename,'w')
    f.write('# GEOMETRY:   CARTESIANn')
    f.write(str(len(b)-1)+'n')

    for i in range(len(c)):
        f.write(str(int(c[i]))+'  '+str(b[i])+'  '+str(b[i+1])+'n')
    f.write(str(len(b)-1)+'n')
    for i in range(len(c)):
        f.write(str(int(c[i]))+'  '+str(b[i])+'  '+str(b[i+1])+'n')
    f.write(str(len(b)-1)+'n')
    for i in range(len(c)):
        f.write(str(int(c[i]))+'  '+str(b[i])+'  '+str(b[i+1])+'n')

#    f.write('1n')
#    f.write('1 0.0 1.0')
    f.close()

    return '空间构造完成!!!'

#========================================================================

if __name__=='__main__':

    print('开始构建背景!!!')
    width = 128
    index = 1.0
    u     = 1.3
    rho_constant = 0.21*u
#    sw    = ['/public/home/zmf/results/SW128_perpendicular_conduction/',10]    #wdir,number,r
    sw    = ['../SW1/',8]
    clump = [200,10,1.0,50.0]             #number,r,index,e
    mag   = 3.2                           #widthi,widthj
    total = combine(['sw'],'W51C.fits','rho0.dbl',width,index,rho_constant,sw,clump,mag)
    grid('grid0.out',30,width)
    print(ti.asctime())

  \end{lstlisting}



\chapter{进阶代码}
\label{Further}

\section{Mayavi使用}
\label{Mayavi}

\begin{lstlisting}
def temp(wdir):
    D = pp.pload(30,w_dir=wdir)

    I = pp.Image()
    flux=D.prs*1.67e-7/D.rho/1000000/1.3806488e-23
#    print flux.shape
#    flux= (flux-np.mean(flux))*5+np.mean(flux)*5.3
#    flux=nd.gaussian_filter(flux,sigma=(4,4),order=0)
    I.pldisplay(D, flux,x1=D.x1, \
                x2=D.x2,label1='l offset (pc)',label2='b offset (pc)',                                    \
                title='Temperature',
                cbar=(True,'vertical'))
#    savefig('MHD_Blast.png') # Only to be saved as either .png or .jpg
    plt.show()

def td(ty,t,E,rho,sigma,wdir):
    D = pp.pload(t,w_dir=wdir)

    print(D.x1.shape)
#        arr = np.meshgrid(D.x1,D.x2,D.x3)
#        mlab.points3d(arr[0][0:256:8,0:256:8,0:256:8], arr[1][0:256:8,0:256:8,0:256:8], arr[2][0:256:8,0:256:8,0:256:8], D.rho[0:256:8,0:256:8,0:256:8])
    vol = mlab.pipeline.volume(mlab.pipeline.scalar_field(np.log10(D.prs*D.rho)))
    ctf = ColorTransferFunction()
    ctf.add_hsv_point(-8, 0.8, 1, 1)
    ctf.add_hsv_point(-6.5, 0.45, 1, 1)
    ctf.add_hsv_point(-5.4, 0.15, 1, 1)

    vol._volume_property.set_color(ctf)
    vol._ctf = ctf
    vol.update_ctf = True
    otf = PiecewiseFunction()

    otf.add_point(-8, 0)
    otf.add_point(-5.7, 0.082)
    otf.add_point(-5.4, 0.0)

    vol._otf = otf
    vol._volume_property.set_scalar_opacity(otf)
#        mlab.contour3d(D.prs)
#    mlab.quiver3d(D.bx1, D.bx2, D.bx3)
#    src = mlab.pipeline.vector_field(D.bx1, D.bx2, D.bx3)
#    mlab.pipeline.vectors(src, mask_points=20000, scale_factor=30.)
#    mlab.outline()
#        mlab.savefig(str(t)+'.obj')

#==================================main========================================
if __name__=='__main__':
    #font = {'family' : 'serif',
    #        'weight' : 'normal',
    #        'size'   : 12,
    #        }

    choose = 'single' #single or multiple or temp
    t      = 50
    E      = 1.3
    rho    = 0.21
    sigma  = 4

    wdir = './'
    nlinf = pp.nlast_info(w_dir=wdir)
    td(ty,t,E,rho,sigma,wdir)

\end{lstlisting}


\section{并行画图}
\label{mpi4py}

\begin{lstlisting}

import numpy as np
from mpi4py import MPI
import time
import pyPLUTO as pp

import naima
from multiprocessing import Pool
from naima.models import (ExponentialCutoffPowerLaw, Synchrotron,
                          InverseCompton)
from astropy.constants import c
import astropy.units as u
#==============================================================================
def f(rho, B):
    ECPL = ExponentialCutoffPowerLaw(1e36*rho*u.Unit('1/eV'), 1*u.TeV, 2.0, 13*u.TeV)
    SYN = Synchrotron(ECPL, B=1000*B*u.uG)

    # Define energy array for synchrotron seed photon field and compute
    # Synchroton luminosity by setting distance to 0.
    Esy = np.logspace(-6, 6, 100)*u.eV
    Lsy = SYN.flux(Esy, distance=0*u.cm)

    # Define source radius and compute photon density
    R = 2 * u.pc
    phn_sy = Lsy / (4 * np.pi * R**2 * c) * 2.26

    # Create IC instance with CMB and synchrotron seed photon fields:
    IC = InverseCompton(ECPL, seed_photon_fields=['CMB', 'FIR', 'NIR',
                                                  ['SSC', Esy, phn_sy]])

    # Compute SEDs
    spectrum_energy = np.logspace(-8,14,10)*u.eV
    sed_IC = IC.sed(spectrum_energy, distance=1.5*u.kpc)

    return sed_IC.value
#==============================================================================
#n = 1 * u.cm**-3
#l = 1 * u.pc
#print((n*l**3).to(''))
#==============================================================================
if __name__ == '__main__':
    t      = 10
    sigma  = 1.0
    vindex = 0.0
    i      = 2

    wdir = './'
    nlinf = pp.nlast_info(w_dir=wdir)

#==============================================================================
    D = pp.pload(t,w_dir=wdir)
    xlabel = 'x (pc)'
    ylabel = 'y (pc)'

\end{lstlisting}
