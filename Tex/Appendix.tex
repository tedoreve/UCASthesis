\chapter{工具型代码}
\label{Code}

\section{单位制计算}
\label{Codeu}


\begin{lstlisting}
from astropy import units as un
from astropy import constants as con
import numpy as np

#====================单位计算====================

UNIT_DENSITY = 1*con.m_p/un.cm**3
UNIT_LENGTH  = 1*un.pc
UNIT_VELOCITY= 1e4*un.km/un.s
UNIT_B = (UNIT_VELOCITY*np.sqrt(4*np.pi*UNIT_DENSITY)).to(un.g**0.5*un.cm**-0.5*un.s**-1).value*un.G
UNIT_t = UNIT_LENGTH/UNIT_VELOCITY
UNIT_P = UNIT_DENSITY*UNIT_VELOCITY**2
UNIT_M = UNIT_DENSITY*UNIT_LENGTH**3
UNIT_E = UNIT_M*UNIT_VELOCITY**2
#UNIT_B = ((UNIT_E/UNIT_LENGTH**3)**0.5).value*un.G
UNIT_NU= UNIT_P*UNIT_t
UNIT_G = (UNIT_VELOCITY/UNIT_LENGTH)**2/UNIT_DENSITY

#===============输入需要转换的参量================

n   = 0.21*con.m_p/un.cm**3
l   = 4*un.pc
v   = 490*un.km/un.s
B   = 9*un.uG
t   = 1000*un.yr
P   = 1*un.Ba
E_th= 0.96*un.erg
E   = 2.0e51*un.erg
M   = 15.9*con.M_sun
nu  = 2*un.uPa*un.s
G   = 1*con.G

#=================开始转换========================

n   /= UNIT_DENSITY
l   /= UNIT_LENGTH
v   /= UNIT_VELOCITY
B   /= UNIT_B
t   /= UNIT_t
P   /= UNIT_P
E   /= UNIT_E
M   /= UNIT_M
nu  /= UNIT_NU
G   /= UNIT_G

#================输出结果=========================

print('n = ', n.to('').value, '\n'
      'l = ', l.to('').value, '\n'
      'v = ', v.to('').value, '\n'
      'B = ', B.to('').value, '\n'
      't = ', t.to('').value, '\n'
      'P = ', P.to('').value, '\n'
      'E = ', E.to('').value, '\n'
      'M = ', M.to('').value, '\n'
      'nu = ', nu.to('').value, '\n'
      'G = ', G.to('').value, '\n'
      )

\end{lstlisting}


\section{构造初始背景}
\label{Codeb}

\begin{lstlisting}

import numpy as np
import time as ti
from astropy.io import fits

#=====================加入星风==================

def stellar_wind(wdir,number):
    import pyPLUTO as pp
    pp.nlast_info(w_dir=wdir)
    D = pp.pload(number,w_dir=wdir)
    print(D.rho.shape)

    rho = D.rho
    bx1 = D.bx1
    bx2 = D.bx2
    bx3 = D.bx3
    vx1 = D.vx1
    vx2 = D.vx2
    vx3 = D.vx3

    rho = np.transpose(rho)
    bx1 = np.transpose(bx1)
    bx2 = np.transpose(bx2)
    bx3 = np.transpose(bx3)
    vx1 = np.transpose(vx1)
    vx2 = np.transpose(vx2)
    vx3 = np.transpose(vx3)

    return rho,bx1,bx2,bx3,vx1,vx2,vx3

#==================加入磁场====================

def toff(f):
    def wrapper(*args):
        start = ti.time()
        f(*args)
        end   = ti.time()
        print(end-start)
    return wrapper

def f(i,j,k):
    return i*2+j*2

#@toff
def magnetism(width):
    x = np.fromfunction(f,(width,width,width))/500000
    x = np.rot90(x,k=1)
    x = np.transpose(x)
    x = np.reshape(x,width**3,1)
    x = x + 0.001
    return x*0, x, x

#=================组合背景=====================

def combine(components,infilename,outfilename,width,index,rho_constant,sw,clump,mag):
    if 'sw' in components:
        wdir,number = sw
        rho,bx1,bx2,bx3,vx1,vx2,vx3    = stellar_wind(wdir,number)

        rho       = np.reshape(rho,(1,width**3))
        bx1       = np.reshape(bx1,(1,width**3))
        bx2       = np.reshape(bx2,(1,width**3))
        bx3       = np.reshape(bx3,(1,width**3))
        vx1       = np.reshape(vx1,(1,width**3))
        vx2       = np.reshape(vx2,(1,width**3))
        vx3       = np.reshape(vx3,(1,width**3))

        total     = np.concatenate((rho,bx1,bx2,bx3,vx1,vx2,vx3))

    total     = total.astype(float)
    total.tofile(outfilename)

    return total

#==================网格定义========================

def grid(outfilename,ra,width):
    b=np.linspace(-ra,ra,width+1)
    c=np.linspace(1,width,width)
    f=open(outfilename,'w')
    f.write('# GEOMETRY:   CARTESIANn')
    f.write(str(len(b)-1)+'n')

    for i in range(len(c)):
        f.write(str(int(c[i]))+'  '+str(b[i])+'  '+str(b[i+1])+'n')
    f.write(str(len(b)-1)+'n')
    for i in range(len(c)):
        f.write(str(int(c[i]))+'  '+str(b[i])+'  '+str(b[i+1])+'n')
    f.write(str(len(b)-1)+'n')
    for i in range(len(c)):
        f.write(str(int(c[i]))+'  '+str(b[i])+'  '+str(b[i+1])+'n')

#    f.write('1n')
#    f.write('1 0.0 1.0')
    f.close()

    return '空间构造完成!!!'

#========================================================================

if __name__=='__main__':

    print('开始构建背景!!!')
    width = 128
    index = 1.0
    u     = 1.3
    rho_constant = 0.21*u
#    sw    = ['/public/home/zmf/results/SW128_perpendicular_conduction/',10]    #wdir,number,r
    sw    = ['../SW1/',8]
    clump = [200,10,1.0,50.0]             #number,r,index,e
    mag   = 3.2                           #widthi,widthj
    total = combine(['sw'],'W51C.fits','rho0.dbl',width,index,rho_constant,sw,clump,mag)
    grid('grid0.out',30,width)
    print(ti.asctime())

  \end{lstlisting}
