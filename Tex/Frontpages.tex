%---------------------------------------------------------------------------%
%->> 封面信息及生成
%---------------------------------------------------------------------------%
%-
%-> 中文封面信息
%-
\confidential{}% 密级:只有涉密论文才填写
\schoollogo{scale=0.095}{ucas_logo}% 校徽
\title{复杂星周环境中超新星遗迹的磁流体模拟}% 论文中文题目
\author{张孟飞}% 论文作者
\advisor{田文武~研究员~张海燕~研究员}% 指导教师:姓名 专业技术职务 工作单位
\advisors{中国科学院国家天文台}
\degree{博士}% 学位:学士、硕士、博士
\degreetype{理学}% 学位类别:理学、工学、工程、医学等
\major{天体物理}% 二级学科专业名称
\institute{中国科学院国家天文台}% 院系名称
\date{2019~年~6~月}% 毕业日期:夏季为6月、冬季为12月
%-
%-> 英文封面信息
%-
\TITLE{Magnetohydrodynamics Simulation for Supernova\\Remnants in
Complex Circumstellar Environment}% 论文英文题目
\AUTHOR{Zhang Mengfei}% 论文作者
\ADVISOR{Supervisor: Profs. TIAN Wenwu \& ZHANG Haiyan}% 指导教师
\DEGREE{Doctor}% 学位:Bachelor, Master, Doctor。封面格式将根据英文学位名称自动切换,请确保拼写准确无误
\DEGREETYPE{Philosophy}% 学位类别:Philosophy, Natural Science, Engineering, Economics, Agriculture 等
\THESISTYPE{Dissertation}% 论文类型:thesis, dissertation
\MAJOR{Astrophysics}% 二级学科专业名称
\INSTITUTE{National Astronomical Observatories of China\\Chinese Academy of Sciences}% 院系名称
\DATE{June, 2019}% 毕业日期:夏季为June、冬季为December
%-
%-> 生成封面
%-
\maketitle% 生成中文封面
\MAKETITLE% 生成英文封面
%-
%-> 作者声明
%-
\makedeclaration% 生成声明页
%-
%-> 中文摘要
%-
\chapter*{摘\quad 要}\chaptermark{摘\quad 要}% 摘要标题
\setcounter{page}{1}% 开始页码
\pagenumbering{Roman}% 页码符号

超新星遗迹(SNR)是超新星爆发后与周围星际介质相互作用的产物,其产生过程涉及天体物理、粒子物理、
分子化学等多个领域,是目前天文研究中的前沿课题之一。通过观测SNR并分析其中元素丰度、动力学特性,我们
可以估算其前身星类型,为超新星爆发模型提供参考;同时基于其多波段能谱(SED),我们可以剖析其中粒子加速
机制,为宇宙线的起源找到更加合理的解释;而对其分子谱线的观测研究,也让我们能进一步理解星际介质和
星系中分子生成与破坏的过程。这些过程都与超新星遗迹的演化密不可分。 最近几十年以来,对SNR演化过程及对
周边物质影响的研究有了更深入的了解,但没解决的难点问题依然很多,尤其对SNR在不同环境里的演化细节和
物理机制知道得还很不充分。对SNR更多新的观测和深入的理论探讨依然是非常必要的。

研究超新星遗迹演化主要的困难在于众多SNR所处物理环境很不一样,单一的解析模型无法解释很多遗迹的观测结果。
因此,我们需要一个基于简洁明了的理论,能梳理清楚复杂情况的方法来研究超新星遗迹的演化,而磁流体
模拟正是最佳选择。磁流体模拟的重点在于对初始条件的选取,可因为初始条件实质上就是过去发生的事,
所以理论上我们永远无法得知一个遗迹的准确初始条件。实际上,有一些爆发时就观测到的
历史超新星遗迹可以得到较为可靠的爆发能量及抛射质量,尤其是SN 1987A,可以说是研究超新星遗迹演化的
范本。但是这些遗迹数量很少,不具有普适性,对于银河系中已经观测到的300多个遗迹,绝大部分前身星
的性质并不清楚。此外,超新星遗迹的演化受到星周介质的巨大影响,而通常星周介质初始的
密度、磁场等性质更难得到。于是,很多对超新星遗迹的模拟都假设均匀、线性或者幂律的星周介质分布,
这与实际复杂的介质分布相差甚远,所以也很难解释有不同表象的观测结果。

本文尝试寻找更好的方法解决这两方面的困难,主要的工作有:
\begin{itemize}
  \item 基于对SNR的X射线观测得到的元素丰度、电子密度估计,结合射电波段对超新星遗迹的HI测距,估算较为可靠前身星
  爆发能量、抛射物质量,结合磁流体模拟,互相验证参数估计。
  我们选择观测研究较多的SNR W51C为例,以方便检验模拟结果。
  前人已经粗略估算出其前身星质量和爆发能量,可是我们经过初步模拟发现,这样的参数无法模拟这颗遗迹,
  同时也违背了最近通过观测得到的超新星爆发质量-能量的经验模型。
  通过检验,我们发现前人估算爆发能量使用的距离并不合理,最终我们建议使用更合理的距离4.3 kpc和爆发能量
  1.3 $\times$ 10$^{51}$ ergs s$^{-1}$。

  \item 为了得到SNR W51C的单边厚壳层,我们采取了普遍使用的设置磁场梯度的方法,偶然发现了SNR W51C
  应该存在一个新的壳层。
  超新星爆发时如果星周介质存在一个磁场梯度,那么最终演化结果可能会是一个单壳层的超新星遗迹。
  而SNR W51C一直被认为是一个单壳层的遗迹,而我们的模拟结果显示其存在一个可能暗弱但仍然能够被观测
  到的壳层。
  通过查看射电图像我们判断,这一新壳层很可能已被观测到,但因为与周围电离氢区在视线方向重合,所以一
  直没有被分离出来,而后来对该SNR射电偏振图像的分析证实了我们的猜测。

  \item SNR W51C同时也是一个少见的TeV伽马射线源,被认为与分子云(MC)相互作用。
  我们模拟了其沿视线方向与分子云相互作用,结果显示这种相互作用可能导致局部磁场放大,射电流量增强,
  从而可能在遗迹的中心区域观测到射电辐射。
  而对临近SNR W51C中心区域的射电偏振、非热辐射及羟基(OH)谱线图的分析,证实了这种相互作用方式的
  存在。

  \item 我们认为一个大质量恒星晚期的星风对其爆发为超新星时的星周介质有很大影响。
  通过模拟星风,并以星风模拟结果作为星周介质的初始条件,我们很好地模拟了超新星遗迹演化,以此为
  基础我们能够解释很多之前无法解释的遗迹形态,比如八字形和大弧度单边遗迹。

  \item 以考虑星风的超新星遗迹模拟为基础,我们认为前身星星风可以大大影响遗迹局部的磁场及介质分布。
  前人曾经通过观测超新星遗迹的磁场来估算银河系大尺度的磁场分布,而我们的这一工作表明,这种估算磁场
  的方式有严格限制条件,不然会出现问题。

  \item 通过模拟SNR在强磁场环境中的演化,我们发现多壳层遗迹的形成、奇特的SNR射电和X射线轮廓、费米泡
  的形成或许都与强磁场有关。
\end{itemize}

本文在模拟中采用更为合理的复杂星周环境,从而更准确地研究SNR演化。
根据结果,我们也得到很多之前没有注意到的演化特征,以及由此而来的推论。
不仅解释了广受关注的SNR W51C中存在多年的问题,也首次通过模拟星风-超新星遗迹系统来解释
SNR的多种射电形态,同时首次研究了强磁场极端条件下的SNR演化。
这些工作对我们进一步理解SNR演化过程具有极大参考价值。

\keywords{星际介质:超新星遗迹 - 磁流体模拟 - 星际介质:SNR W51C - 方法:数值模拟}% 中文关键词
%-
%-> 英文摘要
%-
\chapter*{Abstract}\chaptermark{Abstract}% 摘要标题

Supernova remnant (SNR) results from the interaction between shock wave of supernova
and circumstellar medium (CSM), a process related to astrophysics, particle physics and molecule
chemistry, etc.
To study SNRs and its interaction with the CGM is a frontier project in astronomical
researches.
By observing an SNR and analyzing its abundance and the kinematics, we can derive
the type of its progenitor, which will provide indication for models of supernova explosion.
Meanwhile, based on its spectral energy distribution (SED) at multiband, we can disentangle
the intrinsic acceleration mechanism, and understand the origin of cosmic rays better.
In addition, the spectral observation to an SNR can help us understand the formation
and destruction of molecule in ISM of a galaxy.
To study SNRs may help to figure out these issues, the detail how an SNR is involving has
not been well understood.

The real picture of the SNR's evolution is complex in evolution of SNR, so it is difficult to study
the details by using a traditional analytical model.
Therefore, we need a model-based method to get the situation sorted.
The magnetohydrodynamics (MHD) simulation is the best choice for our case.
The key point of the MHD simulation is to choose initial conditions.
Initial conditions are actually the past things, so we theoretically will never know
the accurate initial conditions of the evolution of an SNR without perfect history record.
The reliable initial conditions of some historical SNRs may be estimated if they have well
been monitored before its explosion.
For example, SN 1987A is known as a gift of God.
Nevertheless, there are only several historical SNRs so that we are not able to use them to explain
most of about 300 SNRs already detected in Milky Way.
Moreover, the evolution of SNRs is strongly influenced by CSM, which leads
 more difficult to estimate the initial density and magnetic field of CSM.
Many simulations for SNRs are assumed to have homogeneous, liner, power-law or
exponential density distribution, which is completely different from the reality, so we fail in
using these simulation to  explain many observational phenomena.

In this paper, we describe the better method to solve the issues mentioned above. The primary
results are shown as follows:
\begin{itemize}
   \item By using the estimation of abundance and electron density based on X-ray observations
   of an SNR and the distance of an SNR by the HI observation, we can estimate the explosion
   energy and ejecta mass of its progenitor.
   We may further check the parameters by comparing observations and simulations.
   We choose the well-studied SNR W51C as an example, so that we can test whether we perform the
   simulation correctly.
   Previous researchers have estimated its progenitor mass and explosion energy, but we cannot
   successfully simulate this SNR by applying such parameters.
   We find these parameters are also inconsistent with that from the derived mass-energy model
   based on recent observation.
   We therefore analyze the methods and parameters carefully and finally take a recently-measurement
   distance of 4.3 kpc and an explosion energy of 1.3 $\times$ 10$^{51}$ ergs s$^{-1}$ as the
   initial parameter of W51C.

   \item To simulate the thick unilateral shell, we set a gradient of magnetic field, a method
   often used to obtain such shell.
   By analyzing the results, we find there should be a new shell of SNR W51C.
   A supernova will evolve to an SNR with unilateral shell, if there exists a magnetic field
   gradient.
   SNR W51C presents a clear semi-shell in previous study, but our simulation shows there is
   possibly another new edge which is much darker but should be detected by previous observations
   with enough sensitivity.
   We investigate its radio images and conclude this new edge overlaps with a radio-strong HII
   region.
   Further analysis aiming at radio polarization map also confirms our result.

   \item SNR W51C is also a TeV $\gamma$-ray source, suggested previously to interact with molecular
   clouds (MCs).
   We simulate how it interacts with an MC along the line of sight (LoS). We find that this process
   can amplify the magnetic field and enhance local radio flux.
   As a result, we can detect radio radiation next to the center of this SNR.
   Meanwhile, we study the radio polarization, non-thermal radiation and OH spectral map in this
   region, which supports such an interaction.

   \item We think the stellar wind of a massive progenitor will largely change the circumstellar
   magnetic field and density distribution.
   We simulate the stellar wind, and take the result as initial conditions for SNR simulation,
   which can help us obtain more accurate simulation.
   This simulation shows a clue to explain many unsolved radio morphologies of SNRs, such as
   bilateral asymmetric and unilateral large-radian SNRs.

   \item Based on the simulation of stellar wind, we notice the stellar wind of progenitor will
   strongly influence local magnetic field.
   However, some researchers try to estimate large-scale magnetic field distribution by
   observing the magnetic field of SNRs, which will need be carefully treated if our simulation
   is correct.

   \item By simulating the evolution of SNRs in strong magnetic field, we notice
   the formation of multi-layers SNRs, the strange radio and X-ray patterns of SNRs
   and the formation of Fermi bubbles are all possibly related to strong magnetic field.
\end{itemize}

    In this paper, we apply reliable complex circumstellar environment in the simulations,
    so that we can study the evolution of SNRs accurately.
    Based on these simulations, we find many ignored features during the evolution and
    deduce some interesting conclusions.
    We solve some problems related to a well-known SNR W51C, and firstly explain radio
    morphologies of various SNRs by simulating a system simultaneously including stellar wind
    and supernova remnants.
    In addition, we study SNRs evolving in strong magnetic field for the first time.
    These works are very important for understanding the evolution of SNRs.

\KEYWORDS{ISM: supernova remnants – magnetohydrodynamics (MHD) -
ISM: individual objects (W51 C) - method: simulations}% 英文关键词
%---------------------------------------------------------------------------%
