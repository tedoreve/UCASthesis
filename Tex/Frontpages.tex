%---------------------------------------------------------------------------%
%->> 封面信息及生成
%---------------------------------------------------------------------------%
%-
%-> 中文封面信息
%-
\confidential{}% 密级:只有涉密论文才填写
\schoollogo{scale=0.095}{ucas_logo}% 校徽
\title{复杂星周环境中超新星遗迹的磁流体模拟}% 论文中文题目
\author{张孟飞}% 论文作者
\advisor{田文武~研究员}% 指导教师:姓名 专业技术职务 工作单位
\advisors{}% 指导老师附加信息 或 第二指导老师信息
\degree{博士}% 学位:学士、硕士、博士
\degreetype{理学}% 学位类别:理学、工学、工程、医学等
\major{天体物理}% 二级学科专业名称
\institute{中国科学院国家天文台}% 院系名称
\date{2019~年~6~月}% 毕业日期:夏季为6月、冬季为12月
%-
%-> 英文封面信息
%-
\TITLE{Magnetohydrodynamics (MHD) Simulation for Supernova\\Remnants (SNRs) in
Complex Circumstellar Environment}% 论文英文题目
\AUTHOR{Zhang Mengfei}% 论文作者
\ADVISOR{Supervisor: Professor Tian Wenwu}% 指导教师
\DEGREE{Doctor}% 学位:Bachelor, Master, Doctor。封面格式将根据英文学位名称自动切换,请确保拼写准确无误
\DEGREETYPE{Philosophy}% 学位类别:Philosophy, Natural Science, Engineering, Economics, Agriculture 等
\THESISTYPE{Dissertation}% 论文类型:thesis, dissertation
\MAJOR{Astrophysics}% 二级学科专业名称
\INSTITUTE{National Astronomical Observatories of China\\Chinese Academy of Sciences}% 院系名称
\DATE{June, 2019}% 毕业日期:夏季为June、冬季为December
%-
%-> 生成封面
%-
\maketitle% 生成中文封面
\MAKETITLE% 生成英文封面
%-
%-> 作者声明
%-
\makedeclaration% 生成声明页
%-
%-> 中文摘要
%-
\chapter*{摘\quad 要}\chaptermark{摘\quad 要}% 摘要标题
\setcounter{page}{1}% 开始页码
\pagenumbering{Roman}% 页码符号

超新星遗迹(SNR)是超新星爆发后与周围星际介质相互作用的产物,其产生过程涉及天体物理、粒子物理、
分子化学等多个领域,是目前天文研究中的前沿课题之一。通过观测分析其中元素丰度、动力学特性,我们
可以估算其前身星类型,为超新星爆发模型提供参考;同时基于其多波段能谱(SED),我们可以剖析其中粒子加速
机制,为宇宙线的起源找到更加合理的解释;而对其分子谱线的观测研究,也让我们能偶进一步理解星系中
分子生成与破坏的过程。这些问题都与超新星遗迹的演化密不可分,可是其具体演化过程其实直到现在也并不
清楚。

研究超新星遗迹演化主要的困难在于具体的情况非常复杂,单一的解析模型无法解释很多遗迹的观测结果。
因此,我们需要一个基于简洁明了的理论,又能梳理清楚复杂情况的方法来研究超新星遗迹的演化,而磁流体
模拟正是最佳选择。磁流体模拟的重点在于对初始条件的选取,可因为初始条件实质上就是过去发生的事,
所以理论上我们永远无法得知一个过去未被观测到的遗迹的准确初始条件。实际上,有一些爆发时就观测到的
历史超新星遗迹可以得到较为可靠的爆发能量及抛射质量,尤其是SN 1987A,可以说是研究超新星遗迹演化的
范本。但是这些遗迹数量很少,不具有普适性,对于银河系中已经观测到的300个左右的遗迹,大部分其前身星
的性质并不清楚。此外,更为棘手的是,超新星遗迹的演化受到星周介质的巨大影响,而通常星周介质初始的
密度、磁场等性质更难得到。于是,很多对超新星遗迹的模拟都基于均匀星周介质分布,或者假设线性、幂律、
指数型的分布,这与实际复杂的介质分布相差甚远,所以也解释不了很多观测结果。

本文尝试寻找更好的方法解决这两方面的困难,主要的工作有:
\begin{itemize}
  \item 基于X射线的元素丰度、电子密度估计,结合射电波段对超新星遗迹的HI测距,估算较为可靠前身星
  爆发能量、抛射物质量,结合磁流体模拟,互相验证参数估计。
  我们选择观测研究较多的SNR W51C为例,以方便检验模拟结果。
  前人已经粗略估算出其前身星质量和爆发能量,可是我们经过初步模拟发现,这样的参数无法模拟这颗遗迹,
  同时也违背了最近通过观测得到的超新星爆发质量-能量的经验模型。
  通过检验,我们发现其估算爆发能量使用的距离并不合理,最终我们建议使用更合理的距离4.3 kpc和爆发能量
  1.3 $\times$ 10$^{51}$ ergs s$^{-1}$。

  \item 为了得到SNR W51C的单边厚壳层,我们采取了普遍使用的设置磁场梯度的方法,偶然发现了SNR W51C
  应该存在一个新的壳层。
  超新星爆发时如果星周介质存在一个磁场梯度,那么最终演化结果可能会是一个单壳层的超新星遗迹。
  而SNR W51C一直被认为是一个单壳层的遗迹,而我们的模拟结果显示其存在一个可能暗弱一些单仍应该被观测
  到的壳层。
  通过查看射电图像我们认为,这一新壳层是因为与周围电离氢区在视线方向重合,所以一直没有被注意到,
  而后来对射电偏振图像的分析也证实了我们的猜测。

  \item SNR W51C同时也是一个伽马射线源,被认为与分子云(MC)相互作用。
  我们模拟了其沿视线方向与分子云相互作用,结果显示这种相互作用可能导致局部磁场放大,射电流量增强,
  从而可能在遗迹的中心区域观测到射电辐射。
  而对临近SNR W51C中心区域的射电偏振、非热辐射及羟基(OH)谱线图的分析,证实了这种相互作用方式的
  存在。

  \item 我们认为一个大质量恒星晚期的星风对其最终成为超新星时的星周介质有很大影响。
  通过模拟星风,并以星风模拟结果作为星周介质的初始条件,我们更准确地模拟了超新星遗迹演化,以此为
  基础我们可以解释很多之前无法解释的遗迹形态,比如八字形和大弧度单边遗迹。

  \item 以考虑星风的超新星遗迹模拟为基础,我们认为前身星星风可以大大影响遗迹局部的磁场及介质分布。
  前人曾经通过观测超新星遗迹的磁场来估算银河系大尺度的磁场分布,而我们的这一工作表明,这种估算磁场
  的方式有的时候会出现问题。
\end{itemize}

本文不仅涉及天体物理的研究,也包括大量具体磁流体模拟的理论及程序编写的内容。其中,磁流体模拟理论
主要介绍一个模拟中需要考虑的各种因素,以及如何找到最适合解决某个问题的模拟方式。而程序编写部分包
括对个人用过的磁流体模拟软件的介绍、编写逻辑及编写过程中可能遇到的问题。

\keywords{星际介质:超新星遗迹 - 磁流体模拟 - 星际介质:SNR W51C - 方法:数值模拟}% 中文关键词
%-
%-> 英文摘要
%-
\chapter*{Abstract}\chaptermark{Abstract}% 摘要标题

Supernova remnant (SNR) is the result of interaction between shock wave of supernova
and circumstellar medium (CSM), a process related to astrophysics, particle physics and molecule
chemistry, etc.
This is one of the most forward projects in astronomical researches.
By observing and analyzing the abundance and the kinetic nature in an SNR, we can derive
the type of its progenitor, which will provide reference for models of supernova explosion.
Meanwhile, based on its spectral energy distribution (SED) at multiband, we can disentangle
the acceleration mechanism inside, and explain the origin of cosmic rays more reasonably.
In addition, the observations on its molecular spectra, can help us understand the formation
and destruction of molecule in a galaxy.
The evolution of SNR is the key to figure out these subjects, but the detailed evolution is
still ambiguous.

The practical situation is complex in evolution of SNR, so it is difficult to study the details
and an analytical model cannot well explain many observations.
Therefore, we need a model-based method that can get the situation sorted, and magnetohydrodynamics
(MHD) simulation is the best choice.
The key point of MHD simulation is to choose initial conditions.
However, initial conditions are actually the past things, so we theoretically will never know
the accurate initial conditions of the evolution of an SNR undetected in the past.
In fact, we can estimate reliable initial conditions of some historical SNRs, if we observed them
carefully when they exploded as supernovae.
In particular, SN 1987A is even known as a gift of universe in this area.
Nevertheless, there are only few historical SNRs which cannot be used as samples to explain
most of about 300 SNRs in Milky Way.
Moreover, the evolution of SNRs is strongly influenced by CSM, but it
is more difficult to estimate the initial density and magnetic field of CSM.
As a result, many simulations for SNRs are assumed to have homogeneous, liner, power-law or
exponential density distribution, which is completely different from the reality, so they
cannot explain many observational phenomena.

In this paper, we try to find better method to solve the problems in the two aspects. The primary
results are shown as follows:
\begin{itemize}
   \item Based on the estimation of abundance and electron density from X-ray data, we can combine
   the HI distance measure to estimate the explosion energy and ejecta mass of progenitor.
   Then we can check the parameters by comparing observations and simulations.
   We choose the well-studied SNR W51C as an example, so that we can know whether we perform the
   simulation correctly.
   Some researchers have estimated its progenitor mass and explosion energy, but we cannot
   simulate this SNR by applying such parameters.
   We find these parameters also disobey the derived mass-energy model based on recent observation.
   After checking their method, we think the distance used to estimate the explosion energy is not
   reasonable.
   Finally, we suggest a distance of 4.3 kpc and an explosion energy of 1.3 $\times$ 10$^{51}$ ergs
   s$^{-1}$.

   \item To simulate the thick unilateral shell, we set a gradient of magnetic field, a method
   often used to obtain such shell.
   By analyzing the results, we find there should be a new shell of SNR W51C.
   A supernova will evolve to an SNR with unilateral shell, if there exists a magnetic field
   gradient.
   SNR W51C is believed to only have one edge, but our simulation shows there is possibly
   another new edge which is much darker but should be detected by previous survey.
   We investigate its radio images and think this new edge overlaps with a surrounding HII
   region, which is why we do not notice it before.
   Further analysis on radio polarization map also confirm our hypothesis.

   \item SNR W51C is also a $\gamma$-ray source, known to interact with molecular clouds (MCs).
   We simulate it interacts with an MC along the line of sight (LoS), which shows this process
   can amplify the magnetic field and enhance local radio flux.
   As a result, we can detect radio radiation next to the center of this SNR.
   Meanwhile, we study the radio polarization, non-thermal radiation and OH spectral map in this
   region, which confirms such an interaction.

   \item We think the stellar wind of a massive progenitor will largely change the circumstellar
   magnetic field and density distribution.
   We simulate the stellar wind, and take the result as initial conditions for SNR simulation,
   which can help us obtain more accurate simulation.
   This simulation shows a clue to explain many unsolved radio morphologies of SNRs, such as
   bilateral asymmetric and unilateral large-radian SNRs.


   \item Based on the simulation of stellar wind, we notice the stellar wind of progenitor will
   strongly influence local magnetic field.
   However, some researchers try to estimate large-scale magnetic field distribution by
   observing the magnetic field of SNRs, which will be a little unreasonable if our simulation
   is correct.
\end{itemize}

   In this paper, we do not only focus astrophysics, but also involve MHD theories and practical
   coding.
   We here introduce various factors we need to consider in a simulation and explain how to find
   suitable simulation method for one particular problem.
   In addition, we also introduce the features of different codes, the logic of coding and some
   unexpected problems.

\KEYWORDS{ISM: supernova remnants – magnetohydrodynamics (MHD) -
ISM: individual objects (W51 C) - method: simulations}% 英文关键词
%---------------------------------------------------------------------------%
