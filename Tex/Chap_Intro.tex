\chapter{引言}
\label{Intro}

\section{超新星遗迹简介}
\label{SNRintro}
超新星遗迹(Supernova Remnants, SNR)是超新星爆发后,其抛射物与周围介质相互作用的产物。
广义的超新星遗迹还包括中心可能残留的致密天体,不过在本文中不予考虑。
第一个有记录可循的观测到的超新星遗迹是蟹状星云(Crab Nebula),这个遗迹是1054年一次超新星
爆发的残骸,虽然超新星爆发本身在当时就被多国天文学家观测到,但是遗迹是过了近700年,才被
约翰·贝维斯和查尔斯·梅西耶分别独立发现\citep{barrow2011cosmic}。
第二个被发现的是开普勒超新星遗迹(Kepler's Supernova Remnant)\citep{Baade1943},这两个
遗迹是射电观测开始之前,唯二的实际被观测到的遗迹\citep{Minkowski1964},两次发现相隔
近200年,这主要是因为SNR在光学波段的辐射受到严重消光的影响,很难观测到。
射电观测开始之后,我们首先发现了直至今日仍是研究重点的第谷超新星遗迹(Tycho's
Supernova Remnant),而随着X射线和$\gamma$射线观测的增多,我们得以在多波段观测到越来越多
的超新星遗迹。

\subsection{流体演化模型}

SNR的演化过程主要受到前身星和周围星际介质(Interstellar Medium, ISM)这两者的影响。
而其实在爆发之前,其前身星的星风如果很强,星风与ISM相互作用会产生性质不太一样的星周介质
(Circumstellar Medium, CSM),早期SNR的演化多少会受到CSM的影响,中晚期的遗迹尺度已经
超出CSM的范围,其演化主要与ISM有关。
超新星爆发时,其抛射物速度很高,与周围介质作用时会产生很强的激波,然后在激波面形成很薄
的壳层。
一开始激波面温度很高,也有很高的压强,但是随着速度降低,辐射耗散会导致其温度下降。
当速度下降到与介质随机运动速度相当,温度也与介质温度相似,我们就无法辨认这个遗迹了。

SNR在均匀介质中的演化理论在很早的时候就已经发展得比较成熟
\citep{1959sdmm.book.....S, 1970IAUS...39..229W, Woltjer1972},直到今天我们也基本
沿用其中的假设及推导。
通常,如果我们认为超新星爆发于均匀的相对冷的介质(设CSM与ISM相同且温度为银河系平均介质温度)
中时,要从SNR中推导超新星爆发的总能量$e_0$,主要计算其中的热能和动能即可。
如果再忽略引力、加速粒子非线性效应和磁场影响,我们就可以构建比较经典的演化模型。
这个模型中,设抛射物总质量为$M_0$,平均介质密度为$\rho_0$,激波半径为$R$,激波速度$V$,
单位时间辐射能量损失为$de/dt$,我们将SNR的演化分为四个阶段:

\begin{enumerate}

    \item 自由膨胀相($M_0\gg(4\pi/3)\rho_0R^3$):这个阶段中扫过的介质对演化的影响非常小,
    SNR演化主要受到爆发过程影响。

    \item 绝热膨胀相(Sedov相,$M_0\ll(4\pi/3)\rho_0R^3$且$\int (de/dt) dt\ll e_0$):
    这个阶段中,扫过的介质对演化起重要作用,但是辐射耗散可忽略。
    这里要注意的是,这里的激波仍然是无碰撞激波,波前的挤压不是通过物质相互碰撞,而是
    激波区磁场加速导致,虽然也可类比为一种碰撞,但是这种碰撞并不会交换热量。
    而整个系统通过粒子碰撞与外界交换热量的时标远远大于激波扫过交换区的时标,因而是绝热的,
    这也是这个阶段名字的由来。
    \citet{1959sdmm.book.....S}第一次得到了在这种情况下的相似解,因而又名为Sedov相。
    这里如果绝热指数$\gamma$=5/3,可得

    \begin{equation}
      \begin{aligned}
        R = 1.17\left(\dfrac{e_0}{\rho_0}\right)^{1/5}t^{2/5},
      \end{aligned}
    \end{equation}

    因而可得

    \begin{equation}
      \begin{aligned}
        V = \dfrac{dR}{dt} = \dfrac{2}{5}\times1.17\left(\dfrac{e_0}{\rho_0}\right)^{1/5}t^{2/5-1} = \dfrac{2}{5}\dfrac{R}{t}.
      \end{aligned}
    \end{equation}

    由此可见$R^3V^2$是一个常数,又因为这个阶段介质密度$\rho_0$是均匀的,
    所以其动能是守恒的。
    而这个阶段总能量只考虑动能和热能,因辐射耗散不重要,所以热能也是守恒的。

    \item 辐射相($\int (de/dt) dt\approx  e_0$):这个阶段中,辐射耗散不可忽略,但是遗迹
    壳层速度仍然比周围介质平均运动快一些,或者来自遗迹的辐射任强,故遗迹仍然可见。

    \item 消失相:遗迹壳层速度与周围介质平均运动相当,遗迹辐射很弱,故遗迹已然不能跟周
    围介质区分开了,基本无法却分。

\end{enumerate}

在这里要提醒的是,其实两个阶段之间很难找到绝对的分界点,此外,这个理论成立的一些假设都是
简化的理想情况。
首先是均匀介质假设,这几乎是不可能的,尤其是前身星的星风总会对周围介质有一定影响,不要说
介质本身肯定也不是均匀的。
更特殊的情况是,SNR周围刚好有致密的分子云(Molecular Cloud, MC),膨胀的壳层刚好与分子云
相互作用,这会很大程度上影响SNR的演化\citep{1972PhDT........47S, Orlando2008, Zhang2017},
而我们在章节~\ref{W51C}中也谈到了这种情况。
至于介质冷的假设在恒星形成区可能问题很大,尤其是大质量恒星经常形成于活跃的HII区,
周围介质温度很可能较高,此外,这种区域更有可能有分子云,也不利于上一条均匀假设。
还有,不考虑引力虽然在大多数情况下是没问题的,但是对于核坍缩型超新星留下的遗迹演化到年老阶段,
如中心致密天体又速度很高可能会冲出壳层时,引力就会影响周围介质的分布。
此外,粒子在激波区域会得到加速,较高的速度会导致粒子逃逸,最终我们的能量守恒就不再适用了。
不过这种情况更多的出现在辐射相和消失相,类似于辐射耗散,在前两相中可以忽略
\citep{1967IAUS...31..117K}。
最容易出现问题是磁场微弱假设,没有磁场,我们几乎不可能在射电波段看到SNR,可是传统做法却在
流体演化的模型中忽略了磁场,这是一个需要正视的问题。
通常,星际磁场的确较弱,无法对遗迹的演化产生根本影响,但是在激波壳层与介质相互作用过程中,
磁场放大是非常明显的,所以近几年的研究工作其实都会考虑磁场的影响。
特别是,有一些区域,星际磁场本身就很强。

我们在做具体模拟的时候,对超新星爆发区域的初始条件设置是通过分析解求得,但是这里的计算
其实是不太实用的。
所以,真实使用的分析解是根据\citet{Truelove1999}的工作求得的,同时与\citet{Leahy2017a}
写的计算器所求结果作比较,如果没问题,再写在初始条件里。

\subsection{观测特征}
对蟹状星云和开普勒超新星遗迹的证认最初是在光学波段,不仅与历史超新星位置符合,而且还
观测到了光学的壳层。
这种光学观测随后得到很好的延续,尤其是随着光谱仪器的发展,我们能从谱线观测得到更多关于
SNR的信息。
SNR中比较明显的光谱特征就是很强的禁线辐射,强度接近$H_\alpha$,而普通的行星状星云中
辐射强度只有$H_\alpha$的10\%。

上世纪中叶后,对SNR的观测主要是在射电波段。
超新星遗迹与介质相互作用会加速电子至相对论能量同时放大磁场,而相对论电子在磁场中运动会
产生同步辐射,在射电波段有很大的流量,我们会在章节~\ref{TheoryDSACR}详细解释这个机制。
因为这个射电辐射机制的特殊性,SNR的能谱(Spectral Energy Distribution, SED)通常都是
规则的幂律谱。
而早期的射电观测显示,SNR在1 GHz波段的面亮度$\Sigma$貌似与遗迹的直径$D$有一个线性关系,
这个关系其实也与这个辐射机制有关,而且在早期是一个估计SNR距离的重要手段。
此外,如果周围介质密度、磁场是均匀的,超新星爆发都是球对称的,那么SNR的形状应该都是规则的,
而我们也根据形状将其分为壳层型、实心型、复合型。
当然其实很多遗迹的形态非常复杂\citep{West2016},很多人都有尝试解释这件事\citep{Zhang2018},
迄今仍然还有很多未解的谜题。
此外,这种机制产生的辐射带有很强的偏振,其偏振度与磁场直接相关,而越来愈多的偏振观测证明
SNR中的磁场远超银河系平均磁场强度。

上世纪80年代后,对X射线辐射随着空间卫星的观测也有了越来越多的新发现,SNR的X射线辐射主要
可以分为热辐射和非热辐射两类,涉及韧致、同步及逆康普顿三种辐射机制,其中的谱线辐射更是
对我们了解遗迹的元素丰度及证认前身星提供了巨大帮助。
这里的热辐射主要是热等离子体的韧致辐射和谱线辐射,而非热辐射其实不仅来自于同步辐射和逆康普顿
辐射,也可能来自非热电子的韧致辐射。
此外,有时遗迹中心会有X射线点源,这是我们了解与其相关遗迹前身星的重要依据之一。
在X射线波段辐射机制的复杂性也导致了有时需要用多个模型拟合观测结果,与其他波段结合一起研究,
在X射线波段的研究发现SNR中存在100TeV电子产生的辐射,同时电子的能量很难超过1PeV
\citep{Reynolds1999}。

最近,对SNR的高能$\gamma$射线观测逐渐受到大家的重视,也正是这些观测提供了SNR中粒子
加速的直接证据。
目前,$\gamma$射线的SED主要还是幂律谱,这应该主要是逆康普顿辐射的产物,有时高能电子的韧致
辐射也会有所贡献。
随着其角分辨率的提高,我们已能观测到更多$\gamma$射线辐射的精细结构,比如\citet{Abeysekara2018}
确实观测到了一个SNR中心致密天体喷流产生的高能辐射,几乎可以肯定这个产生喷流的致密天体是
一个黑洞(Black Hole, BH),这也是银河系中第一个也是目前唯一一个得到证认的SNR-BH系统。

虽然,几乎在全波段都有对SNR的观测,但是更多的观测仍然非常重要,尤其是超低频射电
和中微子观测。
目前对这两种信使的观测还很少,希望将来能有更进一步的发展。

\section{SNR磁流体模拟简介}
\label{MHDintro}
早期的经典流体模型因为有很多简化,实际与观测有很多不相符的地方。
而最容易出现问题的简化就是没有考虑磁场,可是磁场演化的非线性导致很难得出可靠的分析解。
近来,计算机的发展已经逐渐成熟,进行磁流体(Magnetohydrodynamics, MHD)模拟是一个
有望解决这一问题的手段。

早在上世纪70年代,已经有人尝试进行相关模拟\citep{1978MmSAI..49..513S},但是因为计算能力
的限制,无法得到很好的结果。
不过相关领域发展迅速,到了上世纪末,已经可以对SNR演化中的磁场放大、不稳定
性、湍流形成等进行较细致的模拟和解释\citep{Jun1996, Jun1996a, Jun1999}。
最近几年,更高分辨率的模拟帮助我们更清楚地理解SNR的形态特征\citep{Orlando2007},甚至
非均匀介质中的演化\citep{Ferreira2008}。
更关注局部的模拟让我们了解了遗迹中湍动对磁场放大和非热辐射的影响\citep{Fang2014, Ji2016b},
同时也能用来研究特殊遗迹中的偏振辐射\citep{Schneiter2015, Velazquez2017},
更进一步的,粒子加速和高能辐射的结构也可以在模拟中得到体现\citep{Yang2015, vanMarle2018}。

在硬件发展的同时,很多专业的MHD模拟程序逐渐得到优化,为初学者提供了更方便的接口。
在一开始,主要是ZEUS一批二维模拟的软件,而后ZEUS-3D、Athena++、PLUTO、FLASH逐渐丰富模拟
内容,增加各种易用的工具,并提供各种交流渠道,使得在这一领域的研究变得入门容易,深入也有迹可循
\citep{Stone1992b, Stone1992, Stone1992a, White2016, Mignone2007, Mignone2012, Fryxell2000}。
我们虽然在这篇文章中主要介绍PLUTO,不过Athena++和FLASH也是各有优势。
Athena++是最近开发的软件,使用最新的技术,从而使得整个模拟变得简洁高效。
而FLASH是一个历史很长的软件,支持的物理非常丰富,而且维护完善,如果要考虑辐射转移和引力,
这将是最好的选择。
PLUTO本身提供很多简单易懂的测试代码,对初学者比较友好,而且最近刚更新的版本支持计算粒子加速对演化
的影响,这对超新星遗迹的模拟是非常重要的。

\section{论文的目标和主要内容}
\label{Aim}
超新星遗迹其实与很多物理相关,最相关的课题如星系元素丰度组成、尘埃的形成与破坏和粒子在激波
中的加速,比较远的如中心黑洞或中子星与SNR的相互作用、SNR对分子云的挤压激发坍缩等,
每一个问题都需要较长时间的知识积累才能解决。

本文主要通过介绍超新星遗迹的磁流体模拟,希望对复杂星周环境中的粒子加速机制和辐射过程有更
深入的理解。同时,通过详细阐述模拟的具体物理含义和程序编写,对整个研究的实践操作进行概括。

具体内容如下:第二章介绍超新星遗迹模拟中宇宙线产生机制和磁流体模拟理论基础,第三章
详述PLUTO进行流体模拟的程序编写和一些工具使用,第四章是对特定超新星遗迹模拟以及与观测
结合进行讨论的一个解释性工作,第五章是对超新星遗迹前身星星风进行模拟的一个构建模型的工作,
第六章是对SNR在强磁场环境中的演化进行模拟的工作,第七章是对博士研究课题的总结。
