\chapter{总结}
\label{Sum}

我们在本文中首先介绍了超新星遗迹的流体演化模型和辐射特性,并详细阐述了扩散激波加速机制的
具体理论和局限,并以此为基础说明宇宙线的起源和SNR中的各种辐射机制。
经典扩散激波加速机制的重要局限之一就是没有考虑磁场,所以我们接着讨论了磁场在激波加速中的
重要性,并尝试得出了考虑磁场的DSA模型,并借此引出我们在实际模拟中需要知道的一些的磁流体特
性的讨论。
然后我们希望,本文可以帮助他人了解具体磁流体模拟编程的细节,因而以比较流行的模拟软件PLUTO
为例介绍了软件使用中须知的各种参数、不同文件的编写以及一些新技术的使用,同时提供了示例代码
供大家参考。

最后我们展示了超新星遗迹在各种不同星周环境中的磁流体模拟,主要结论如下:

\begin{enumerate}

    \item SNR W51C在东北方存在一个新的壳层,之前因为W51A辐射遮挡而未被发现。

    \item 电离氢区G49.2-0.35中的非热辐射和偏振辐射来自于W51C与视线方向上分子云相互作用。

    \item 电离氢区G49.2-0.35附近的OH吸收主要来自于G49.2-0.35,而与W51C无关。

    \item 大质量前身星的星风对超新星遗迹的射电形态有很大影响,可能比最初的周围环境影响
    还大。

    \item 将星风考虑在内,我们可以解释很多遗迹的射电形态,除了多层和不规则遗迹。

    \item 根据模拟结果,我们不建议通过超新星遗迹的射电图像推测大尺度的磁场和密度分布。

    \item 热传导或许会稍微影响遗迹的射电形态,但是并不是非常重要。

    \item 某些遗迹的射线壳层与X射线辐射的偏离可能是因为前身星的运动。

    \item 强磁场中反向激波行为在不同方向存在很大差异,形成类似于喷流准直的现象。

    \item 这种准直现象在银河系中心或许会影响到费米泡的形成。

    \item 强磁场中演化的SNR也更有可能出现奇怪的射电和X射线图像,这或许可以解释
    G1.9+0.3的形态。

    \item 多壳层超新星遗迹在强磁场环境中容易出现,可实际形成机理与很多因素相关。

\end{enumerate}

这些工作中遇到的问题也很多,有的需要观测进一步证实,有的则需要更高分辨率、更大尺度的模拟,
有的需要很多次调参测试。
我们下一步比较倾向于研究费米泡是否可以用SNR演化来解释,这个相比其它问题而言,更多是计算
性能的限制,如果能使用更高性能的集群将会提供很大帮助。
