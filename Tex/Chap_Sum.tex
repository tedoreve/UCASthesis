\chapter{总结与展望}
\label{Sum}

我们在本文中首先介绍了超新星遗迹的流体演化模型和辐射特性,并详细阐述了扩散激波加速机制的
具体理论和局限,并以此为基础阐述了宇宙线的起源和SNR中的各种辐射机制。
经典扩散激波加速机制的重要局限之一就是没有考虑磁场,所以我们接着讨论了磁场在激波加速中的
重要性,并尝试得出了考虑磁场的DSA模型,并借此描述了我们在实际模拟中需要知道的一些磁流体特
性。
本文也给出了具体磁流体模拟编程的细节,以比较流行的模拟软件PLUTO
为例介绍了软件使用中需要设置的各种参数、不同文件的编写以及一些新技术的使用,同时提供了示例代码
供大家参考。

博士期间科学上最主要的贡献是对超新星遗迹在各种不同星周环境中完成了磁流体模拟,并将模拟结
果与具体观测相结合。主要结论如下:

\begin{enumerate}

    \item 预测并证实SNR W51C在东北方存在一个新的壳层。

    \item 电离氢区G49.2-0.35中的非热辐射和偏振辐射来自于W51C与视线方向上分子云相互作用。

    \item 电离氢区G49.2-0.35附近的OH吸收主要来自于G49.2-0.35,而与W51C无关。

    \item 大质量前身星的星风对超新星遗迹的射电形态有很大影响,可能比最初的周围环境影响
    还大。

    \item 将星风考虑在内,我们可以解释很多遗迹的射电形态,除了多层和不规则遗迹。

    \item 根据模拟结果,我们不建议通过超新星遗迹的射电图像推测大尺度的磁场和密度分布。

    \item 热传导或许会稍微影响遗迹的射电形态,但是并不是非常重要。

    \item 某些遗迹的射线壳层与X射线辐射的偏离可能是因为前身星的运动。

    \item 强磁场中反向激波行为在不同方向存在很大差异,形成类似于喷流准直的现象,这种准直
    现象在银河系中心或许会影响到费米泡的形成。

    \item 强磁场中演化的SNR也更有可能出现奇怪的射电和X射线图像,这有助于解释
    G1.9+0.3的形态。

    \item 多壳层超新星遗迹在强磁场环境中容易出现,可实际形成机理与很多因素相关。

\end{enumerate}

总之,博士期间完成了一些有趣课题,用一个新的解释解决了围绕SNR W51C的一系列问题,成功
构建星风-超新星遗迹系统解释了SNR的射电形态,并模拟SNR在强磁场中的演化以解释一些奇特的
观测现象。
在模拟W51C时,我们根据结果预言了一些W51C的性质,而这些都被后来的数据分析一一证实,这是
在一个工作中完成了一整套完整的科学实践过程:根据观测建立理论,根据理论做模拟,将模拟结果
与观测比较从而完善理论。
而在模拟星风-超新星遗迹系统时,我们首次模拟了星风对SNR射电演化的影响,建立了系统的SNR
射电形态形成模型。
之后,我们首次模拟SNR在比银河系平均磁场强100倍的磁场中的演化,这个模拟为我们完善之前建立
的射电形态形成模型提供了支持,同时给出费米泡形成的一种新机制。

但工作中遇到的问题也很多,有的需要观测进一步证实,
有的需要很多次调参测试,有的则需要更高分辨率、更大尺度的模拟。
我们在W51C模拟中预言电离氢区G49.2-0.35方向经过足够长的时间后,会产生更多脉泽点,而且
非热辐射和偏振辐射都会增强,这需要较长时间的观测跟踪来证实。
我们可以对这个区域隔一段时间观测一次,如果出现新的脉泽点,则证明预言正确,如果灵敏度足够高,
较短时间内或许就可以观测到非热辐射和偏振辐射的增强。
此外,我们希望理解多壳层遗迹的形成原因,这需要在参数空间做多次测试,我们希望在将来编写代码,
实现参数调试的自动化,从而更高效的估算不同参数对多壳层的形成贡献。
我们下一步倾向于研究费米泡是否可以用SNR演化来解释,这个问题能否回答好科学上很有关注度,
同时相比其它问题而言,更多是计算性能的限制,在银河系中心分辨率要足够高从而保证SNR初期演化
的准确性,同时模拟尺度要足够大与费米泡尺度相当,将来使用更高性能的集群将会提供很大帮助。
