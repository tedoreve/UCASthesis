\chapter{作者简历及攻读学位期间发表的学术论文与研究成果}


\section*{作者简历}

\subsection*{基本情况}

张孟飞,山东省青岛市人,中国科学院国家天文台博士研究生。

\subsection*{教育经历}

2010年9月至2014年6月,北京师范大学天文系,本科,天文学。

2014年9月至2019年6月,中国科学院国家天文台,直博生,天体物理。

\section*{已发表(或正式接受)的学术论文:}

[1] \textbf{Zhang, M. F.}, Tian, W. W., Wu, D. (2018, October). How does the stellar wind influence the radio morphology of a supernova
remnant? ApJ, 867. doi:10.3847/1538-4357/aae090. arXiv: 1810.03777

[2] \textbf{Zhang, M. F.}, Tian, W. W., Leahy, D. A., Zhu, H., Cui, X. H., Shan, S. S. (2017, November). Disentangling the Radio Emission of the
Supernova Remnant W51C. ApJ, 849. doi:10.3847/1538-4357/aa901d. arXiv: 1710.04770

[3] Su, H. Q., \textbf{Zhang, M. F.}, Zhu, H., Wu, D. (2017, September). The revised distance of supernova remnant G15.4+0.1. Research in
Astronomy and Astrophysics, 17. doi:10.1088/1674-4527/17/10/109. arXiv: 1707.04188

[4] Zhu, H., Tian, W., Li, A., \textbf{Zhang, M. F.} (2017, November). The gas-to-extinction ratio and the gas distribution in the Galaxy.
MNRAS, 471. doi:10.1093/mnras/stx1580. arXiv: 1706.07109

[5] Shan, S. S., Zhu, H., Tian, W. W., \textbf{Zhang, M. F.}, Zhang, H. Y., Wu, D., Yang, A. Y. (2018, October). Distances of Galactic Supernova
Remnants Using Red Clump Stars. ApJS, 238. doi:10.3847/1538-4365/aae07a. arXiv: 1810.06014

[6] Liu, W., Zhu, M., Dai, C., Wang, B.-Y., Wu, K., Yu, X.-C., Tian, W.-W., \textbf{Zhang, M.-F.}, Wang, H.-F. (2019, March). A deep learning
approach for detecting candidates of supernova remnants. Research in Astronomy and Astrophysics, 19, 042. doi:10.1088/1674-
4527/19/3/42

[7] Wu, D., \textbf{Zhang, M. F.} How does the strong surrounding magnetic field influence the evolution of a supernova remnant? Research
in Astronomy and Astrophysics, (Accepted).

[8] Wu, D., \textbf{Zhang, M. F.}, Shan, S. S., Tian, W. W. (2017, February). MHD Simulation of Supernova Remnants. In A. Marcowith, M.
Renaud, G. Dubner, A. Ray, A. Bykov (Eds.), Supernova 1987a:30 years later - cosmic rays and nuclei from supernovae and their
aftermaths (Vol. 331). IAU Symposium. doi:10.1017/S1743921317004902

[9] Shan, S. S., Wu, D., Zhu, H., \textbf{Zhang, M. F.}, Tian, W. W. (2017, February). Measuring distances to Galactic SNRs using the red
clump stars. In A. Marcowith, M. Renaud, G. Dubner, A. Ray, A. Bykov (Eds.), Supernova 1987a:30 years later - cosmic rays and
nuclei from supernovae and their aftermaths (Vol. 331). IAU Symposium. doi:10.1017/S1743921317004914

\section*{参加的研究项目及获奖情况:}

获得2016-2017年度AMD奖学金。

获得2015-2016年度中国科学院大学三好学生荣誉称号。

\chapter[致谢]{致\quad 谢}\chaptermark{致\quad 谢}% syntax: \chapter[目录]{标题}\chaptermark{页眉}
\thispagestyle{noheaderstyle}% 如果需要移除当前页的页眉
%\pagestyle{noheaderstyle}% 如果需要移除整章的页眉

值此论文完成之际,谨在此感谢读博期间一直支持我、鼓励我的家人、老师、同学和朋友。

首先,感谢我的导师田文武研究员,他对我的宽容与信任,使得我可以自由地学习、研究、探索,让我懂得了
真正的科研精神。
他的悉心指导、循循善诱与无比的耐心,让我在探求新知识的同时,能够脚踏实地,习得科研的基本素养。
他真正做到了因材施教,甚至比我自己还了解我,在我迷惘时总能帮我找到方向,甚至是我自己都不相信的方向。
选择博士课题时,田老师经过与我讨论,建议我做超新星遗迹的磁流体模拟,而当时的我甚至没有学过流体力学。
起初我是发怵的,担心自己无法胜任这一工作,不过我也认识到这是一个极有价值的课题。
于是,我犹豫着,最终还是装作自信满满的听取了田老师的建议。
这种行为让我想起了小时候课堂回答问题,老师提问我不会,但是全班同学都积极地举手,我也装作积极地举手,
然后就被点名了。
真正开始这项课题的时候,感觉一直在被点名,理论知识不懂,编程语言不通,研究前沿不明。
这时,田老师却一直在对别人说我工作做的很好,学习能力很强,勇于开创新的方向,这让我都怀疑自己是
不是真的很强。
他对我的认可,最终战胜了我对自己的不信任,我最终还是把这条路走下来了。
同时,我要感谢他不仅让我学到了科研中的严谨、实践、创新,更重要的是让我学到了一些他善良的内心、
幽默的谈吐以及令人心安的人格魅力,有时我想,如果我无法取得博士学位,能拥有这些品质应该也不枉这求学五年。

再者,我要感谢我的第二导师张海燕研究员,她让我认识到深入的科研是另外一个世界,一个合格的科研
人员必须对自己课题的方方面面都有足够的理解。
本来我想专注做模拟,分析观测数据不需要对仪器了解得那么深入,但与她讨论过FAST接收机等问题后,我意识到工程与
科研的有机结合是非常重要的,我不能只囿于自己的理论模拟,实验证实才是科学前进的关键一步。
另外,也要感谢张老师对本文提出的意见,令文章重点突出,整体鲜明。
同时,我要感谢吴丹老师。
我在这里纠结了好久该称她为老师还是师姐,她于我是亦师亦友,生活中更像我的朋友,而在我心中,她也是我
我的导师之一。
她总是能与他人友好相处,与她交谈的人如沐春风,与她一起工作总是充满活力。
她就像是加在我们整个团组的buff,关怀着每一个人。
而我尤其受益良多,田老师对我的鼓励与信任有时候也是一种压力,而吴丹老师的照顾让我内心释然,能将其转化为一种动力。
我知道自己应该学习她的这种为团队带来能量的品格,可这是多年经历凝结而成的,不是我一朝一夕能够领悟,
我所能做的大概就是今后会尽量帮助身边每一个人。

还有,感谢崔晓红老师经常发给我一些会议、交流、工作的信息,让我能够及时掌握最近的学术动态。
感谢朱辉师兄对我在科研上的指导与建议,带我认识了好多牛人。
感谢苏洪全师兄让我对射电设备有了更清晰的认识,当然也要谢谢他愿意跟我一起瞎聊我提出的逗比想法。
感谢素素师姐经常提醒我各种注意事项,我有时候会不看各种通讯设备错过各种消息,如果不知道可能
会出很大纰漏,还要谢谢素素师姐愿意跟我聊天,一聊大半个下午,从此不用看新闻联播,很爽。
感谢嫒媛师姐分享她的各种经验,办公室有个美女很重要,令人神清气爽,提高工作效率。
感谢天体物理综合研究团组的每一个成员,与他们的相处让我获益良多,受用终生。

感谢Dr. Reich、高旭阳师兄、徐钧师兄对偏振数据的解释,感谢Dr. Rugel和Dr. Beuther对THOR数据的解释,
感谢Dr. Leahy和方军教授对一系列问题的讨论,感谢张祥光教授、李志远教授对我的肯定和支持。
感谢国家天文台帮助过我的各位老师,容忍我的拖延、马虎、失礼,为我的学习生活、人事变动、签证
办理等事项提供便利。

感谢程功、曹烨、侯义军、金云鹏、杨成群在怀柔的时候一起偷吃小火锅,感谢李朝、胡文凯、王珅回到
奥运村这边继续陪我偷吃小火锅。
感谢沈丹、王瑞和所有311的小伙伴一起愉快玩耍,愿我们聚是一团火锅,散是满天星。
感谢赵赫、小优、勇哥、小胖、炳坤、东哥等各位,在我科研不顺利的时候找我来玩,陪我散心,浪费时间,
进度更慢,恶性循环。

最后,感谢父母对我的理解和支持,感谢各位亲友对我父母的陪伴,感谢他们对我从小到大的帮助和照顾。
没有你们的生活安康,我是无法专心科研的。
感谢祖父母虽然一直唠叨我不经常联系,但还是在我离家的时候偷偷塞红包。
感谢所有亲人对我这些年聚少离多的谅解,谢谢!

\cleardoublepage[plain]% 让文档总是结束于偶数页,可根据需要设定页眉页脚样式,如 [noheaderstyle]
