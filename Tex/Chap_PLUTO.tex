\chapter{使用PLUTO进行磁流体模拟的程序编写}
\label{PLUTO}



\section{PLUTO软件简介}
\label{PLUTOintro}
PLUTO是一个有多国科研人员共同完成的流体模拟软件,至今为止一直得到很好的维护,是进行天体
物理相关模拟的可靠软件\citep{Mignone2007,Mignone2012}。
目前,软件只支持Linux/Unix平台运行,最基本的运行只依赖Python2、C、GNU make,几乎是每个
Linux/Unix平台发行版都会自带的。
而进一步的需求依赖不同,要进行并行计算需要安装MPI library,要输出HDF5格式需要安装HDF5
library,要进行自适应网格(Adaptive Mesh Refinement, AMR)模拟需要安装C++、Fortran和
Chombo library,HDF5 library。
当然,如果要直接输出PNG格式图像,需要安装PNG library,但是我们很少这样做,因为输出数组
格式的二进制文件后可以直接用Python读取进行可视化,结果修改更灵活。

其主题文件目录结构包括:

\begin{enumerate}

    \item Config/:包含系统配置文件,软件要根据其中的内容读取依赖的代码的信息,比如系统是否
    已经安装C,HDF5 library等。这个文件夹中原始文件都是一些针对不同系统的模板,代码编译
    之前,需要根据模板自己编写适合自己系统的配置文件。

    \item Doc/:软件文档,包括pdf文档和html文档,两者内容相差不大,pdf文档更为全面,
    html文档更为直观,html文档还有一个好处是其中的代码可以直接复制用来测试。

    \item Lib/:额外需要的运行库的安装位置。比如要进行AMR模拟,需要安装Chombo库,那么
    相应的库最好安在这个,如果安在别的地方需要额外的路径配置。

    \item Src/:程序主目录,除了初始条件设定部分,所有进行具体运算的代码都在这个目录。

    \item Tools/:有用的小工具。我们主要使用其中的pyPLUTO进行可视化。

    \item Test\_Problems/:包含各种各样使用PLUTO的例子。这些事例对于初学者非常友好,根据
    文档可直接运行,对于了解整个软件的使用很有帮助。

\end{enumerate}

而实际进行模拟时基本不需要修改这其中的文件,可以直接在自己新建的文件夹中操作,在程序运行
时设置好环境变量就行。

PLUTO本身可进行的模拟种类非常多,而本章节只以超新星遗迹模拟为例详细介绍MHD模拟方法,
而可视化方法中只介绍Python的使用与优化,更多的内容请见PLUTO的Doc目录中的官方文档。

\section{PLUTO基本使用}
\label{PLUTOuse}
具体下载安装请见http://plutocode.ph.unito.it,而在文档的Quick Start部分有简要的路径
配置,我们只介绍文件编写和运行。
软件安装、配置完成后,我们需要新建工作目录,比如目录SNR/,而要进行一个完整模拟,这个目录
中至少需要三个文件definition.h、init.c、pluto.ini。
其中definition.h主要定义需要的物理及计算方式,init.c定义初始条件,pluto.ini定义需要的
参数,三者名称都可以改变,但是个人建议使用此默认命名,之后交流比较方便。

而大概使用步骤如下:
\begin{enumerate}

    \item definition.h编写完成后,使用python \$PLUTO\_DIR/setup.py,确认使用的参数,然后选择自己
    在Config/中设置好的配置文件,如果使用Ubuntu等Linux系列平台进行非并行运算,选择默认的
    Linux.gcc.defs即可。

    \item init.c编写完成后,使用make编译程序,程序会根据definition.h中设置的参数进行编译,默认会
    生成名为pluto的可执行程序。

    \item pluto.ini编写完成后,使用./pluto运行程序,pluto程序会从pluto.ini中读取相应的数值,然后
    开始运行,运行结果与具体参数设置有关。

\end{enumerate}

\subsection{definition.h的编写}
\subsubsection{基本选项}

PHYSICS选项,定义需要的物理,可选HD(流体)、MHD(磁流体)、RHD(相对论流体)、
RMHD(相对论磁流体),我们主要介绍磁流体模拟,选MHD即可。
MHD模拟内核主要是通过解质量守恒、动量守恒、能量守恒和磁通守恒方程获得结果,需要注意的是,
这里的压强和能量包括磁压和磁能,要根据物态方程获得温度等信息时,要记得与温度有关的压强是
要扣除磁压的。

DIMENSIONS选项,定义模拟的空间维度,可选1、2、3,如果要进行三维模拟,选3即可。
COMPONENTS选项,设置矢量的分量个数,通常三维空间的矢量有三个分量,但有时为了计算方便也
可以约简其中一个分量,比如模拟恒星盘的形成时,大部分物质都在二维平面上运动,但其运动也会
影响第三维上的动量等参数。通常我们取DIMENSIONS=COMPONENTS即可。

GEOMETRY选项,设置使用的坐标系,可选CARTESIAN(笛卡尔)、CYLINDRAICAL(柱坐标)、
POLAR(极坐标)、SPHERICAL(球坐标)。
不同的坐标对不同的积分方式兼容性不同,一半情况下,选用笛卡尔坐标系即可。

BODY\_FORCE选项,可选POTENTIAL(标量势)、VECTOR(矢量引力)、POTENTIAL+VECTOR
(标量+矢量描述)。
标量势计算方便,但是在不够准确,尤其是引力场较复杂的时候。矢量引力计算时间长很多,但更准
确。我个人没有遇到过必须两者都考虑的情况,不过测试时间与只考虑矢量引力差不太多。

COOLING选项,可选POWER\_LAW、TABULATED、SNEq、H2\_COOL、MINEq。
光学薄的情况下,韧致辐射等过程导致的辐射能量损失也会影响MHD模拟结果,这个过程与当地密度、
温度、各种成分的丰度有关,尤其是丰度部分很难估算,因而模型很多。
POWER\_LAW模型是假设辐射损失与$\rho^2T^{\alpha}$成正比,其中$\rho$为密度,T为温度,
$\alpha$为幂律指数。
TABULATED模型是假设辐射损失与$n^2\Lambda(T)$成正比,其中$n$为数密度,$\Lambda$为冷却
方程,而冷却方程又有很多种\citep{2017RMxAA..53..385F}。
SNEq模型主要考虑了中性氢原子的辐射,同时也将16种最常见的谱线辐射考虑在内,比如Ly$\alpha$、
H$\alpha$、HeI、CI、OII、FeII等。
H2\_COOL模型在SNEq模型的基础上考虑了分子氢和离子氢,此外,可根据需要加入一氧化碳(CO)、
羟基(OH)和水分子(H$_2$O)等辐射贡献。
MINEq模型在SNEq模型的基础上考虑了更过谱线辐射,总共达到28种,不过对分子类辐射贡献的计算
支持不好。
超新星遗迹的模拟中,早期自由膨胀相和绝热膨胀相辐射耗散相对总能量很小,对整体模拟结果影响
不大,随着年龄变老,辐射能量越来越大,直到辐射相已经不可忽略。

RECONSTRUCTION选项,可选FLAT、LINEAR、WENO3、LimO3、PARABOLIC,是五种重构方法。
重构是模拟中的一个重要过程,随着时间变化,每一个步长间隔都需要重构一次整个模拟图像。
更好的重构方法可保证模拟结果更加准确,通常使用默认的LINEAR方法即可,如果速度太慢可尝试
FLAT方法,如果设定初始物理参数很极端而结果明显存在问题,可尝试其它三种方法。

TIME\_STEPPING选项,可选EULER、RK2、RK3、CHARACTERISTIC\_TRACING、HANCOCK,是五种
时间推演方法。
要知道,布置邻近格点会影响一个格点下一步的数值,可能更远的格点也会影响,如果步长过长,
会有过多格点影响到一个格点的数值,计算结果容易有偏差,当然,步长过短会导致运行过慢。
所以我们要选择合适的推演方法,通常在超新星遗迹模拟中选择RK2即可。
此外,我们可以通过设定Courant-Friedrichs-Lewy(CFL)数来控制具体方法,具体设定不只与
TIME\_STEPPING有关,还要考虑所选维度及是否使用DIMENSIONAL\_SPLITTING。

DIMENSIONAL\_SPLITTING选项,可选YES或者NO。
有时候,虽然我们模拟的是多维的图像,但是不同维度的影响并不会耦合,结果就是我们可以当作
在不同方向上都做一维的模拟,这时候可选YES。
而超新星遗迹的模拟每个方向上的演化都是互相影响的,所以必须选NO,不然结果很容易出错。

NTRACER选项,这其实是一个扩展选项,最大数目为4。
通常对于特定的PHYSICS守恒量都是一定的,如果要增加新的守恒量就要添加在这里,比如加入
氢原子数目守恒。

USER\_DEF\_PARAMETERS选项,输入需要的自定义参数数目,最大31。
设置初始条件时需要很多参数,比如超新星爆发能量、前身星质量、介质平均密度等,将来调参调的
就是这些,而这里只是提前设置好参数数目。

\subsubsection{物理依赖选项}
BACKGROUND\_FIELD选项,如果设置为YES,意味着设置磁场包含背景的静态场和随时变化的磁场。
超新星遗迹当中的磁场并不需要分成分,都是随时变化的,因而设为NO即可。
可是,如果需要考虑脉冲星风云中几乎保持不变的中子星磁场,就要设置为YES。

DIVB\_CONTROL选项,可选NO、EIGHT\_WAVES、DIV\_CLEANING、CONSTRAINED\_TRANSPORT。
数值方法无法保证磁场无源,我们需要强制控制其散度为零。
超新星遗迹模拟中使用EIGHT\_WAVES即可。

EOS选项,选择需要的物态方程,可选IDEAL、ISOTHERMAL、PVTE\_LAW、TAUB。
超新星遗迹的模拟中,使用IDEAL即可。

ENTROPY\_SWITCH选项,可选NO、SELECTIVE、ALWAYS、CHOMBO\_REGRID。
保证熵增原则,只对EOS为IDEAL和TAUB适用,不过在超新星遗迹模拟中影响不大,设为NO。

RESISTIVITY、VISCOSITY、THERMAL\_CONDUCTION选项,可选NO、EXPLICIT、SUPER\_TIME\_STEPPING,
控制电磁阻力、粘滞相应、热传导影响,都是有关耗散效应,在年轻超新星遗迹模拟中影响不大,
一般设为NO即可,而对老年超新星遗迹需要具体分析。
我们在章节~\ref{SW}有讨论热传导的影响,最终结论只是对星风演化影响很大。

ROTATING\_FRAME选项,使用CYLINDRAICAL(柱坐标)、POLAR(极坐标)、SPHERICAL(球坐标)
时建议选YES。

\subsubsection{自定义参数和常数}
自定义参数即是根据USER\_DEF\_PARAMETERS选项中设置的数目在这里列出相应数量的参数,而其
具体数值主要在pluto.ini中给出,然后由init.c中写的初始条件融入到整个程序里。

而自定义常数主要是一些只在特定模拟中有意义的数值,同时,编写init.c时需要控制的条件也可以
写在这里,比如设置自定义参数ADD\_TURBULENCE为YES或者NO,而init.c中有相应的条件语句从而
实现不同的功能。
其实大部分物理上的常数已经包含在不过这一部分有一个重要的功能就是设定单位制。
我们知道常用的国际单位制、CGS单位制要表达一些比较大的数值时需要科学记数法,可是在进行
模拟时这种记数法是没用的,当数值非常大或者非常小,超过给其分配的计算空间时,会导致计算溢出,
最终模拟结果肯定会有问题。
所以,我们通常需要自己创造一套适合自己模拟的单位制。
通常制定一套单位制需要三个独立基本物理量的单位即可,其它单位可以由此推出,通常大家都选用
长度、质量、时间,类似CGS单位制。
而天文单位中经常过大或过小,而PLUTO又是一个为天文服务的模拟软件,虽然其主程序默认CGS单位制,
实际模板选用的基本物理量单位是:1 $m_p g/cm^3$(密度)、1 AU(长度)、1 \kms(速度)。
这个单位制对于模拟恒星盘、太阳风等非常合适,但是对于模拟超新星遗迹就捉襟见肘了。
所以,在我们模拟遗迹时,选用的基本物理量单位是:1 $m_p g/cm^3$(密度)、1 pc(长度)、
10000 \kms(速度)。
其它一些需要注意的事情在MHD模拟中无需考虑,比如要使用相对论模块,速度单位必须是光速。

而实际物理研究时,我们通常使用CGS单位制,要将模拟结果转化过来,需要一些计算。
最直观的,单位时间即单位长度除以单位速度,
$t_0 = L_0/v_0 = 1 pc/10000 \kms = 3.09\times10^9 s = 98 年$。
下面列出常用参量计算,无下标的即模拟结果中的实际数值,下标为0的即是用我们所选三个基本单位
计算的用在模拟中的其它相应单位,下标为cgs的即在CGS单位制中的实际数值:

\begin{equation}
  \begin{aligned}
    \rho = \dfrac{\rho_{cgs}}{\rho_0},   \
    v = \dfrac{v_{cgs}}{v_0},   \
    p= \dfrac{p_{cgs}}{p_0v_0^2},   \
    B = \dfrac{B_{cgs}}{\sqrt{4\pi\rho_0}v_0^2},   \
  \end{aligned}
\end{equation}
更多转换请见附录~\ref{Codeu}。

\subsubsection{其它经常使用的选项}
上面三部分的选项,在第一步运行setup.py时还有一次更改的几乎,而这部分的选项只能在definitions.h
里修改。

INITIAL\_SMOOTHING选项,设置为YES可以用来数值平滑起伏很大的区域,主要用来消除与网格化
不匹配的边界效应,比如笛卡尔坐标系中的球形。
然而,开启这个选项会大大降低程序运行速度,有时候反而会平滑掉真实的小尺度结构,不建议使用。
如果最终结果需要平滑,最好直接自己手动编写程序优化图像。

WARNING\_MESSAGES选项,设置为YES可以使得程序遇到问题后打出提示。
实际测试中发现,其实大部分问题都不影响程序运行,这个选项打出的问题只是警告(warning),
并不是错误(error)。
问题是,有时警告太多,将真正有用的运行信息遮挡,如果选择了PRINT\_TO\_FILE选项,输出的文件
会非常大,个人不建议使用。

PRINT\_TO\_FILE选项,顾名思义,设置为YES时,程序运行时本来打印在命令行里的信息会打印
到pluto.log文件中。

INTERNAL\_BOUNDARY选项,设置为YES时,可以在init.c中的UserDefBoundary()函数中定义内
边界,可以使得边界内的参数在模拟过程中保持不变。
这个选项在模拟星风、喷流等持续存在的现象时非常有用。

SHOCK\_FLATTENING选项,可选NO、ONED、MULTID。
因为激波区域参数起伏较大,有时会导致程序出错、卡死,为了减少这种起伏保证程序足够稳定,
可以使用这一选项,平滑激波区域各种参数。
ONED是一维平滑,MULTID是多维平滑,效果比一维要好,不过类似于INITIAL\_SMOOTHING,这个
选项会大大减慢运行速度,尤其是多维平滑,几乎不可用。
建议在遇到比较复杂的激波结构或者模拟出现莫名其妙的错误时使用,单个超新星遗迹的模拟不太需要。

CHAR\_LIMITING选项,选择YES以使得重构时直接计算守恒量等特征变量,而不是计算速度、密度、
磁场等初始变量。
这个方式也是很耗时,相当于每一步计算都增加了一些计算量,但是相对来说在参数变化较大甚至
明显不连续以至于平滑都无法做到的地方很有效,此外对熵增原则也可以更好地保证。
不过在超新星遗迹模拟中并不是很必要。

LIMITER选项,可选项很多,主要是RECONSTRUCTION中选择LINEAR时才可使用,主要是对这种重构
方式的各种调试,如果模拟结果莫名其妙,而且其它任何地方都找不到错误时,可以尝试调试一下。
个人只遇到过一次必须改变这一选项的情况,请酌情使用。

ASSIGN\_VECTOR\_POTENTIAL选项,设置为YES后,磁场成分初始化方式改变,可以保证磁场没有
散度。
只在使用特殊算法时有用,一般情况下使用DIVB\_CONTROL足够。

\subsection{init.c的编写}
