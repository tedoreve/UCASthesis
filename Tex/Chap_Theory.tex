\chapter{超新星遗迹和磁流体相关理论}
\label{Theory}

\section{超新星遗迹激波加速和宇宙线}
\label{TheoryDSACR}
二战结束后,很多用于军事的射电观测设备和技术被应用于天文探索,而这段时间观测发现银河系内
存在很多非热射电源。
\citet{1953AZh....30...15S}首次提出这些非热辐射源自于过去爆发却未被看到的超新星爆发
产生的残骸,而辐射机制就是同步辐射。
这一猜测随着观测的增多逐步得到证实,直到今天,射电观测仍然是证认SNR的重要手段,而截至
目前,我们已经观测到近300个SNR\citep{2014BASI...42...47G}。

可是,同步辐射产生SNR的非热谱有一个前提,就是辐射电子是相对论化且幂律分布的
\citep{1970ranp.book.....P}。
\citet{Fermi1949}提出的宇宙线加速机制刚好使得电子可以满足这一前提,同时也提到了超新星
爆发或者SNR演化或许会引起这样的加速,\citet{1953AZh....30...15S}也是以此为依据提出同步
辐射机制产生SNR中的射电辐射。
而Fermi其实更多是想解释观测到的宇宙线能谱,因为当时逐渐受到关注的宇宙线观测显示其能谱是
幂律谱,这就需要一个加速机制解释这件事,所以他只是提出了一个能产生幂律谱的机制,并没有
深入阐述这个机制如何在SNR中起作用。
这主要是个将磁化云块作为磁镜的机制,
很快,很多人注意到这个机制的加速效率很低,不足以解释SNR中的射电辐射流量。



\subsection{扩散激波加速机制}



\section{磁流体模拟基础}
\label{ThoryMHD}
